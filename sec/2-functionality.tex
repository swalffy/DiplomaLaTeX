\subsection{Определение общего функционала приложений}\label{subsec:2-define-functionality}\indent

Исходя из выводов, сделанных в конце первой главы, мы должны определить набор функций, которые будут реализованы в системе приложений.
Поскольку будет разработано не одно приложение, а целая система, то и функционал будет разделён по принадлежности к определённому приложению.

\subsubsection{Общий функционал Rest-сервера}\indent

Разрабатываемое приложение должно реализовывать базовые функции интернет-магазина.
А именно:

\begin{itemize}
    \item Добавление/изменение/удаление продуктов из базы данных;
    \item Возможность назначение скидок определённым пользователям;
    \item Обеспечение механизмов аутентификации;
    \item Возможность формирования заказов на основе товаров, которые клиент положил в свою корзину;
    \item Рассылка e-mail сообщений на основе загруженных шаблонов e-mail сообщений для обеспечения информирование клиентов и менеджеров о состояниях заказов или об объявлениях и акциях;
    \item Управление внутренними файловыми ресурсами приложения;
    \item Реализация сервера изображений, используемых в клиентских приложениях.
\end{itemize}

\subsubsection{Общий функционал Web-клиента}\indent

Разрабатываемое приложение должно состоять из двух модулей:

\begin{enumerate}
    \item Пользовательская часть;
    \item CMS-часть.
\end{enumerate}

Пользовательская часть приложения предназначена для использования потенциальными клиентами интернет магазина и должны предоставлять возможность:

\begin{itemize}
    \item Регистрации и авторизации на ресурсе;
    \item Просмотр информации об организации;
    \item Просмотр категорий товаров, товаров и их характеристик;
    \item Наполнение корзины;
    \item Оформление заказа;
    \item Контакт с менеджером;
    \item Просмотр и изменение личной информации в личном кабинете.
\end{itemize}

CMS-часть предназначена для использования менеджерами и администратором.
Контроль доступа к этой секции осуществляется сервером.
Обычный, анонимный пользователь или пользователь с недостаточным уровнем доступа, не может попасть на данную секцию приложения.
Основные возможности CMS-части:

\begin{itemize}
    \item Просмотр/добавление/изменение/удаление категорий товаров;
    \item Просмотр/добавление/изменение/удаление товаров, а также изменение списка изображение товара;
    \item Просмотр/добавление/изменение информации о зарегистрированных пользователях, а также изменение их скидок;
    \item Формирование и рассылка почтовых сообщений всем клиентам;
\end{itemize}

\subsubsection{Общий функционал Android-клиента}\indent

Поскольку разрабатываемое приложение должно использоваться менеджерами организации, приложение должно обеспечивать возможность авторизации пользователя с помощью установленных на удалённом сервере авторизационных данных. Неавторизованный пользователь не должен иметь возможности получить какие-либо данные из приложения, поскольку это может привести к раскрытию коммерческой тайны.

Авторизованные пользователи должны иметь возможность, в зависимости от уровня доступа:

\begin{itemize}
    \item Просмотр и изменение категорий товаров
    \item Просмотр и изменение полного списка товаров и их детальные страницы;
    \item Просмотр списка заказов, а также детальной информации по каждому из них;
    \item Просмотр, редактирование и добавление пользователей, редактирование их скидок;
    \item Изменение локальной конфигурации приложения;
    \item Формирование отчетов по определённым критериям
\end{itemize}