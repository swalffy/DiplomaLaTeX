\subsection{Определение общего функционала приложений}\label{subsec:2-define-functionality}\indent

Исходя из выводов, сделанных в конце первой главы, необходимо определить набор функций, которые будут реализованы в системе приложений.
Поскольку будет разработано не одно приложение, а целая система, то и функционал будет разделён по принадлежности к определённому приложению.

\subsubsection{Общий функционал REST-сервера}\indent

Разрабатываемое приложение должно реализовывать базовые функции интернет-магазина.
А именно:

\begin{itemize}
    \item добавление/изменение/удаление продуктов из \hyperlink{gloss:db}{БД};
    \item возможность назначения скидок определённым пользователям;
    \item обеспечение механизмов идентификации, аутентификации и авторизации;
    \item возможность формирования заказов на основе товаров, которые клиент положил в свою корзину;
    \item рассылка e-mail сообщений на основе загруженных шаблонов e-mail сообщений для обеспечения информирования клиентов и менеджеров о состояниях заказов или об объявлениях и акциях;
    \item управление внутренними файловыми ресурсами приложения;
    \item реализация сервера изображений, используемых в клиентских приложениях.
\end{itemize}

\subsubsection{Общий функционал веб-клиента}\indent

Разрабатываемое приложение должно состоять из двух модулей:

\begin{itemize}
    \item пользовательская часть;
    \item \hyperlink{gloss:cms}{СMS}-часть.
\end{itemize}

Пользовательская часть приложения предназначена для использования потенциальными клиентами интернет магазина и должны предоставлять возможности:

\begin{itemize}
    \item регистрация, аутентификация и авторизация на ресурсе;
    \item просмотр информации об организации;
    \item просмотр категорий товаров, товаров и их характеристик;
    \item наполнение корзины;
    \item оформление заказа;
    \item контакт с менеджером;
    \item просмотр и редактирование личной информации в личном кабинете.
\end{itemize}

\hyperlink{gloss:cms}{CMS}-часть предназначена для использования менеджерами и администратором.
Контроль доступа к этой секции осуществляется сервером.
Обычный, анонимный пользователь или пользователь с недостаточным уровнем доступа, не может попасть в данную секцию приложения.
Основные возможности \hyperlink{gloss:cms}{CMS}-части:

\begin{itemize}
    \item просмотр/добавление/изменение/удаление категорий товаров;
    \item просмотр/добавление/изменение/удаление товаров, а также изменение списка изображение товара;
    \item просмотр/добавление/изменение информации о зарегистрированных пользователях, а также редактирование размера их скидок;
    \item формирование и рассылка почтовых сообщений всем клиентам;
\end{itemize}

\subsubsection{Общий функционал Android-клиента}\indent

Разрабатываемое мобольное приложение должно использоваться менеджерами организации, поэтому оно должно обеспечивать возможность авторизации пользователя с помощью установленных на удалённом сервере авторизационных данных. Неавторизованный пользователь не должен иметь возможности получить какие-либо данные из приложения, поскольку это может привести к раскрытию коммерческой тайны.

Авторизованные пользователи должны иметь возможность, в зависимости от уровня доступа:

\begin{itemize}
    \item просмотр и изменение категорий товаров;
    \item просмотр и изменение полного списка товаров и их детальные страницы;
    \item просмотр списка заказов, а также детальной информации по каждому из них;
    \item просмотр, редактирование и добавление пользователей, редактирование их скидок;
    \item изменение локальной конфигурации приложения;
    \item формирование отчетов по определённым критериям.
\end{itemize}