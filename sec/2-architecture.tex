\subsection{Проектирование общей архитектуры приложения}\label{subsec:2-design-architecure}\indent

Разрабатываемое приложение является приложением-сервером в клиент-серверной архитектуре.
Как следствие, для данной архитектуры необходимо использовать технологии, которые способствуют эффективной реализации всех поставленных задач.

\addimghere{app-architecture}{1}{Общая архитектура приложения}

Для реализации сервера будет использован REST API подход к реализации архитектуры.

REST API (англ. Representational State Transfer) – архитектурный стиль взаимодействия компонентов клиент-серверного приложения в сети.
Представляет собой согласованный набор ограничений, учитываемых при проектировании распределённой системы.
В определённых случаях это приводит к повышению производительности и упрощению архитектуры.
Такой подход помогает поддерживать несколько клиентских приложений на разных платформах, а также позволяет поддерживать достаточный уровень абстрагированности и масштабируемости.

Внутренняя архитектура rest-сервера и web-клиента будет следовать архитектурному паттерну MVC (Model – View – Controller).
В связи с особенностями платформы, Android-клиент будет реализован при помощи паттерна MVVM (Model – View – ViewModel).

\subsubsection{Архитектурный паттерн MVC}\indent

\addimghere{arch-mvc}{0.5}{Схема архитектурного паттерна MVC}

MVС – архитектурный паттерн проектирование позволяет разделить приложение на 3 связанные части.
Данный паттерн позволяет выделять из больших компонентов части, которые могут быть пере использованы и способствуют параллельной разработке.

Функциональные слои MVC:
\begin{enumerate}
    \item Model – представляет собой слой данных и реагирует на инструкции контроллера, изменяя своё состояние;
    \item View – отвечает за отображение данных модели пользователю, реагируя на изменение модели;
    \item Controller – принимает ввод и преобразует его в инструкции для модели или представления.
\end{enumerate}

\subsubsection{Архитектурный паттерн MVVM}\indent

\addimghere{arch-mvvm}{0.5}{Схема архитектурного паттерна MVVM}

MVVC – архитектурный паттерн проектирование позволяет разделить приложение на 3 отдельные части. Данный паттерн позволяет выделять из больших компонентов части, которые могут быть пере использованы и способствуют параллельной разработке.

Функциональные слои MVVM:
\begin{enumerate}
    \item Model – представляет собой слой данных.
    Обычно являются структурами или простыми Data классами;
    \item View – отвечает за отображение данных модели пользователю, реагируя на изменение модели.
    Является подписчиком на событие изменения значений свойств или команд, предоставляемых ViewModel.
    В случае, если в ViewModel изменилось свойство, она оповещает об этом своих подписчиков;
    \item ViewModel – содержит Model, преобразованную к View, а также команды, которыми может пользоваться View, чтоб влиять на модель.
\end{enumerate}