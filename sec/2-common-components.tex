\subsection{Описание основных компонентов приложений}\label{subsec:2-common-components}\indent

% ## WEB

\subsubsection{Основные компоненты веб-клиента}

К основным страницам разрабатываемого приложения относятся:

\begin{itemize}
    \item главная страница;
    \item каталог;
    \item корзина/Форма оформления заказа;
    \item страница “Товары” \hyperlink{gloss:cms}{CMS}-части;
    \item страница “Пользователи” \hyperlink{gloss:cms}{CMS}-части.
\end{itemize}

Разберём основной функционал, который должны предоставлять данные страницы.

Главная страница – является отправной точкой для пользователя и содержит основную информацию об организации, владеющей интернет магазином.
По сути является Landing-page.

Каталог – содержит список товаров, по категориям, которые отсортированы по наличию и цене.
Каждый товар обладает своим рядом характеристик, а также изображениями.
Некоторые товары могут обладать цветами, в таком случае, в корзину складывается не просто товар, а еще и его цвет.
Изображения каждого из товаров переключается с заданным интервалом.
В случае, отсутствия изображений, блок с изображениями заменяется на fallback-image.

Корзина/Форма оформления заказа – предоставляет возможность изменить кол-во товара и их список, которые будут использованы при оформлении заказа.
Каждый товар обладает рядом осноных характеристик и одним изображением.
Цена за позицию должна изменяться на лету, в зависимости от кол-ва товаров в корзине.
Форма оформления заказа должна поддерживать валидацию введённых данных, перед отправкой запроса на сервер.
В случае успешного оформления заказа, происходит переадресация на главную страницу приложения и очистки локальной корзины.

Страница “Товары” \hyperlink{gloss:cms}{CMS}-части – страница администраторской части приложения, которая доступна только пользователями с уровнем доступа Manager и выше.
Контроль доступа осуществляется сервером.
На данной странице есть возможность просмотра/добавления/редактирования и изменения категорий товаров и товаров.
Товары сгруппированы по категориям.
Имеется возможность быстрой установки информации о том, что товар отсутствует на складе или удалён.
Кроме того, присутствует возможность изменения информации о товаре и его изображения на специальной форме.
Для добавления товара используется отдельная форма.

% ## ANDROID

\subsubsection{Основные компоненты Android-клиента}

К основным страницам разрабатываемого приложения относятся:

\begin{itemize}
    \item cтраница авторизации;
    \item cписок категорий и их продуктов;
    \item cтраница детальной информации о товаре;
    \item cписок заказов и информация о них;
    \item cписок зарегистрированных пользователей, с возможностью детального просмотра информации, а также удалении/добавления.
\end{itemize}

Разберём основной функционал, который должны предоставлять данные страницы.

Контроль доступа осуществляется со стороны сервера.

Страница авторизации – является отправной точкой для пользователя и содержит небольшое приветственное сообщение и поля для ввода авторотационных данных.

Список категорий и их продуктов содержит список товаров, по категориям.
Каждый товар обладает своим рядом характеристик, а также сопровождается изображениями.
Некоторые товары могут обладать цветами.
Изображения каждого из товаров переключются с заданным интервалом.
В случае, отсутствия изображений, блок с изображениями заменяется на fallback-image.
На данной странице есть возможность просмотра/добавления/редактирования и изменения категорий товаров и товаров.
Товары сгруппированы по категориям.
Имеется возможность быстрой установки информации о том, что товар отсутствует на складе или удалён.
Кроме того, присутствует возможность изменения информации о товаре и его изображения на специальной форме.
Для добавления товара используется отдельная форма.
