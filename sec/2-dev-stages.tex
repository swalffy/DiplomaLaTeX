\subsection{Этапы разработки системы приложений}\label{subsec:2-dev-stages}\indent

Для разработки системы приложения, необходимо разбить данный процесс на этапы и поставить ряд задач для каждого из этапов.

Работу над разработкой системы приложений можно разбить на следующие этапы:

\begin{enumerate}
    \item Постановка задачи и анализ требований;
    \item Проектирование общей архитектуры системы приложений;
    \begin{enumerate}
        \item Проектирование \hyperlink{gloss:rest}{Rest-сервера}:
        \begin{itemize}
            \item описание общей архитектуры приложения;
            \item проектирование диаграммы использования;
            \item проектирование схемы \hyperlink{gloss:db}{БД};
            \item описание технологий, используемых в разработке;
            \item описание основных слоёв приложения;
            \item диаграмма классов приложения.
        \end{itemize}
        \item Проектирование Web-клиента:
        \begin{itemize}
            \item описание общей архитектуры приложения;
            \item пороектирование шаблонов основных экранов приложения;
            \item описание технологий, используемых в разработке;
            \item описание основных компонентов приложения;
            \item диаграмма деятельности некоторых процессов в приложении.
        \end{itemize}
        \item Проектирование Android-клиента:
        \begin{itemize}
            \item описание общей архитектуры приложения
            \item проектирование шаблонов основных экранов приложения
            \item описание технологий, используемых в разработке
            \item описание основных компонентов приложения
            \item диаграмма деятельности некоторых процессов в приложении
        \end{itemize}
    \end{enumerate}
    \item Реализация системы приложений:
    \begin{enumerate}
        \item Реализация основных модулей приложений;
        \item Ad hook тестирование разработанной системы приложений как отдельно, так и во взаимодействии.
    \end{enumerate}
\end{enumerate}