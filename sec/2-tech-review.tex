\subsection{Описание технологий, используемых в разработке}\label{subsec:2-tech-review}\indent

% ## REST

\subsubsection{Технологии, использующиеся при разработке REST-сервера}\indent

В качестве языка программирования использован Kotlin.

Kotlin – статически типизированный язык програмиирования, работающий поверх \hyperlink{gloss:jvm}{JVM} и разрабатываемый компанией JetBrains. Имеет возможность компиляции в Javascript, а также ряд других платформ, через инфраструктуру LLVM. Авторы языка, ставили целью создание более лаконичного и типобезопасного, чем Java и более простого чем Scala языка.\cite{kotlin}

К достоинствам относят:
\begin{itemize}
    \item лаконичность языка;
    \item возможность создания расширений для типов, именнованые аргументы и ряд других фич, которые относят к разряду “cинтаксического сахара”;
    \item Kotlin официально поддерживается Google;
    \item полностью совместим с Java;
    \item при работающем проекте на Java, имеется возможность не переписывать всё на Kotlin, а лишь дописывать новый функционал, без нарушения работы в продукте.
\end{itemize}

К недостаткам можно отнести, достаточно малое сообщество разработчиков, однако оно постоянно расширяется.

Основным фрейворком является Spring Boot,который является упрощенной версией фрейворка Spring.

Spring – один из наиболее популярных фрейворков для разработки приложений для Java (на текущий момент заявлено, что Spring полностью совместим с Kotlin).
К основным особенностям фреймворка относят встроенная поддержка \hyperlink{gloss:di}{DI}, которая позволяет придерживаться принципа \hyperlink{gloss:ioc}{IoC}.
Spring помогает свободно разрабатывать полноценные приложения, которые достаточно просто покрываются юнит-тестами.

Spring boot – является упрощенной версией Spring фреймворка.
Spring boot позволяет взять на себя часть рутины связанной с конфигурацией проекта.

Spring security и Spring oauth2 – позволяют контролировать доступ к методам приложения, а также позволяет производить авторизацию и регистрацию пользователей.

Spring Data \hyperlink{gloss:jpa}{JPA} – реализует слой доступа к данным и призван значительно упростить реализацию слоя доступа к данным, сократив усилия на этом этапе и направив в области, которые действительно необходимы.
Достоинства:
\begin{itemize}
    \item поддержка репозиториев, основанных на Spring и \hyperlink{gloss:jpa}{JPA};
    \item поддержка типобезопасных \hyperlink{gloss:jpa}{JPA} запросов;
    \item прозрачный аудит для доменных классов;
    \item поддержка разбивки на страницы;
    \item возможность интеграции собственного кода для доступа к данным.
\end{itemize}

Для сборки проекта и управления зависимостями использован Gradle.

Gradle – открытая система для автоматизации сборки проектов.
Поддерживает инкрементальную сборку и может определять, какая часть древа была обновлена.
Одним из крупнейших преимуществ Gradle по сравнению с другими системами сборки(Maven, Ant и т.д.) является общая гибкость в настройках сборки и каталогов, 
без необходимости следовать ограничениям системы сборки.

Для написания Unit-тестов использована библиотека JUnit, которая является библиотекой для модульного тестирования \hypertarget{gloss:software}{ПО}.
Изначально данная библиотека была разработана для Java языка. Однако Kotlin полностью совместим с Java, поэтому JUnit может использоваться и для написания тестов для языка Kotlin.

Для общей гибкости при написании тестов использованы библиотеки Mockk и Assertj.

В качестве \hyperlink{gloss:db}{БД} использована реляционная \hyperlink{gloss:db}{БД} MySQL. В реляционной \hyperlink{gloss:db}{БД} данные хрянятся в таблицах.
Взаимосвязанные данные могут группироваться в таблицы, а также между таблицами могут быть установлены взаимоотношения.
К безусловным достоинствам данной \hyperlink{gloss:db}{БД} является контроль доступа, масштабируемость.

% ## WEB

\subsubsection{Технологии, использующиеся при разработке Web-клиента}\indent

Для вёрстки web-страниц использован язык разметки HTML. Для предания страницам дизайна, использован CSS.

HTML (от англ. HyperText Markup Language – «язык гипертекстовой разметки») – стандартизированный язык разметки документов во Всемирной паутине.
Большинство веб-страниц содержат описание разметки на языке HTML (или XHTML).
Язык HTML интерпретируется браузерами;
полученный в результате интерпретации форматированный текст отображается на экране монитора компьютера или мобильного устройства.
Текстовые документы, содержащие разметку на языке HTML (такие документы традиционно имеют расширение .html или .htm), обрабатываются веб-браузерам, которые отображают документ в его форматированном виде, предоставляя пользователю удобный интерфейс для запроса веб-страниц, их просмотра (и вывода на иные внешние устройства) и, при необходимости, отправки введённых пользователем данных на сервер.\cite{web-technologies}

CSS (от англ. Cascading Style Sheets – каскадные таблицы стилей) – формальный язык описания внешнего вида документа, написанного с использованием языка разметки.
Преимущественно используется как средство описания, оформления внешнего вида веб-страниц, написанных с помощью языков разметки HTML и XHTML, но может также применяться к любым XML-документам, например, к SVG или XUL. CSS используется создателями веб-страниц для задания цветов, шрифтов, расположения отдельных блоков и других аспектов представления внешнего вида этих веб-страниц. Основной целью разработки CSS являлось разделение описания логической структуры веб-страницы (которое производится с помощью HTML или других языков разметки) от описания внешнего вида этой веб-страницы (которое теперь производится с помощью формального языка CSS).\cite{web-technologies}

Фрейворк для web-приложения – Angular.

Angular – это открытая и свободная платформа для разработки веб-приложений, написанная на языке TypeScript, разрабатываемая командой из компании Google, 
а также сообществом разработчиков из различных компаний.
Предназначена для разработки одностраничных приложений.
Цель использования — расширение браузерных приложений на основе MVC-шаблона, а также упрощение тестирования и разработки.

Фреймворк работает с HTML, содержащим дополнительные пользовательские атрибуты, которые описываются директивами, и связывает ввод или вывод области страницы с моделью, представляющей собой обычные переменные JavaScript.
Значения этих переменных задаются вручную или извлекаются из статических или динамических JSON-данных.

Двустороннее связывание данных в Angular является наиболее примечательной особенностью, и уменьшает количество кода, освобождая сервер от работы с шаблонами.
Вместо этого, шаблоны отображаются как обычный HTML, наполненный данными, содержащимися в области видимости, определённой в модели.
Специальный сервис в Angular следит за изменениями в модели и изменяет раздел HTML-выражения в представлении через контроллер.
Кроме того, любые изменения в представлении отражаются в модели.
Это позволяет обойти необходимость манипулирования DOM и облегчает инициализацию и прототипирование веб-приложений.\cite{web-angular}

TypeScript — язык программирования, представленный Microsoft и позиционируемый как средство разработки веб-приложений, расширяющее возможности JavaScript.
TypeScript является обратно совместимым с JavaScript и компилируется в последний.
Фактически, после компиляции программу на TypeScript можно выполнять в любом современном браузере или использовать совместно с серверной платформой Node.js.
Код экспериментального компилятора, транслирующего TypeScript в JavaScript, распространяется под лицензией Apache.
Его разработка ведётся в публичном репозитории через сервис GitHub.
TypeScript отличается от JavaScript возможностью явного статического назначения типов, поддержкой использования полноценных классов (как в традиционных объектно-ориентированных языках), а также поддержкой подключения модулей, что призвано повысить скорость разработки, облегчить читаемость, рефакторинг и повторное использование кода, помочь осуществлять поиск ошибок на этапе разработки и компиляции, и, возможно, ускорить выполнение программ \cite{web-typescript}.

Angular Material состоит из набора предустановленных компонентов Angular.
В отличие от Bootstrap, предоставляющего компоненты, которые вы можете использовать любым способом, Anglate Material стремится обеспечить расширенный и последовательный пользовательский интерфейс.
В то же время он дает возможность контролировать, как ведут себя разные компоненты.

Material Design – это язык дизайна для веб и мобильных приложений, который был разработан Google.
Material Design упрощает разработчикам настройку \hyperlink{gloss:ui}{UI}, сохраняя при этом удобный интерфейс приложений.

% ## ANDROID

\subsubsection{Технологии, использующиеся при разработке Android-клиента}\indent

Для разработки нативного Android-приложения использован Android-фреймворк использован язык программирования Kotlin.

В разработке использованы элементы из Android Jetpack Architecture Components:
\begin{itemize}
    \item LiveData – хранилище данных, работающее по принципу паттерна Observer, которое умеет определять активность подписчика;
    \item LifeCycle – компонент для удобной работы с LifeCycle Activity;
    \item Android Ktx – функции расширения для стандартной библиотеки Android;
    \item Navigation – компонент облегчающий навигацию между фрагментами Android приложения;
    \item Room – ORM система для SQLLite;
    \item ViewModel – компонент позволяющий корректно обрабатывать состояние фрагмента или активити при изменении состояния (например, при повороте).
\end{itemize}

В качестве \hyperlink{gloss:di}{DI} фреймворка выступает Koin.
Koin – небольшая библиотека для внедрения зависимостей.
В отличии от большей части подобных библиотек, Koin не использует кодогенерацию, проксировани или итроинспецкию.
Из дополнительных плюсов, Koin использует \hyperlink{gloss:dsl}{DSL} и функционал языка Kotlin.
Подразумевается использования с Kotlin, однако, Java тоже может работать вместе с Koin.

OkHttp использован для реализации возможности выполнения сетевых запросов и сетевого взаимодействия.
Эта библиотека обладает полным функционалом для работы с любым \hyperlink{gloss:rest}{REST API}, легко тестируется и настаивается.
