\subsection{Диаграмма деятельности некоторых функций приложения}\label{subsec:2-activity-diagram}\indent

% ## WEB

\subsubsection{Диаграмма деятельности Web-клиента}\indent

Диаграмма деятельности Web-клиента представлена на рисунке \ref{diagram-activity-web_2}
\clearpage
\begin{sidewaysfigure}
    \addimghere{diagram-activity-web_2}{0.87}{Диаграмма деятельности Web-клиента}

\end{sidewaysfigure}
\clearpage
% % \addimghere{diagram-activity-web_2}{1}{Диаграмма деятельности Web-клиента}

Согласно представленной диаграмме, после авторизации пользователя в приложении, в зависимости от роли, которая закреплена за данным пользователем, изменяется список функционала, 
который доступен пользователю.
Так, например, анонимный пользователь, имеет возможность зарегистрироваться/авторизоваться в приложении, в последствии получив одну из ролей, которая закреплена за данным пользователем.
Помимо авторизации и регистрации, пользователь имеет возможность просматривать каталог товаров, наполнять корзину и делать заказ.

В случае, если пользователь обладает уровенем доступа выше уровня "Менеджер", включительно, для него открывается доступ в \hyperlink{gloss:cms}{CMS} часть сайта, 
в которой есть возможность редактировать список товаров, категорий, пользователей и т.д.

% ## ANDROID

\subsubsection{Диаграмма деятельности Android-клиента}\indent

Диаграмма деятельности Android-клиента представлена на рисунке \ref{diagram-activity-android_2}

\clearpage
\begin{sidewaysfigure}
    \addimghere{diagram-activity-android_2}{0.87}{Диаграмма деятельности Android-клиента}

\end{sidewaysfigure}
\clearpage

Диаграмма деятельности Android-клиента имеет схожую структуру и суть с диаграммой Web-клиента, за исключением невозможности работы неавторизованным пользователем 
и невозможностью пользоваться функционалом предназначенным для простого пользователя приложения (например, складывать товары в корзину или оформлять заказ). 
Однако, пользователю Android-приложения доступен весь функционал, доступный менеджерам и администраторам CMS-части Web-клиента. 
