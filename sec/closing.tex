\anonsection{Заключение}\indent

В дипломной работе разработан ряд приложений.
Одни приложения предназначены для использования интернет-магазина потенциальным клиентом магазина.
Другие -  для управления менеджерами интернет-магазином.
Разработанные приложения позволяют клиентам осуществлять просмотр интересующих товаров, оформлять заказ и связываться с менеджерами сайта с помощью формы обратной связи.
Менеджеры имеют возможность редактировать все доступные позиции, управлять их наличием/отображением на сайте, редактировать данные пользователей, предоставлять им специальные скидки.
Система приложений обладает выделенным сервером на котором хранится база данных.

Для достижения цели дипломной работы были решены следующие задачи.
\begin{enumerate}
    \item Проанализирована предметная область.
    \item Спроектирована как общая архитектура ряда приложений, их способ сообщения, так и архитектура каждого приложения отдельно.
    \item Выбраны средства разработки и приведено обоснование данного выбора.
    \item Разработаны \hyperlink{gloss:db}{БД}, \hyperlink{gloss:api}{API} для связи сервера и клиентов, \hyperlink{gloss:ui}{UI} клинтских приложений.
    \item Разработан \hyperlink{gloss:rest}{REST-сервер}, а также Web и Android приложения-клиенты.
    \item Проанализирован и разработан пользовательский интерфейс с учетом основных тенденций и принципов, повышающих его удобство и позволяющий полностью реализовать необходимый функционал.
\end{enumerate}

Предлагаемая разработка является актуальной, так как решения, представленные на рынке, обладают недостаточным функционалом.
Разработанное приложение будет актуальным для людей и компаний заинтересованных создании своего интернет-магазина оптовой торговли.

По дипломной работе имеется статья, материалы которой опубликованы в сборниках конференции в печатном и электронном виде, 
также в сборниках Гродненского Государственного университета <<Наука - 2019>>.
