\subsection{Проектирование БД}\label{subsec:design-db}\indent

Спроектированная диаграмма \hyperlink{gloss:db}{БД} представлена в приложении \ref{addition:rest-database}.

На данной диаграмме, можно выделить несколько видов связей между таблицами:

\begin{enumerate}
    \item <<Один-ко-многим>> – примером такой связи может выступать связь между сущностями "Категория" и "Товар", поскольку, одновременно товар может находиться только в одной категории.
    \item <<Многие-ко-многим>> – реализуется путём применения связей один-ко-многим через промежуточную таблицу. Так, например, сущность "Товар" связана связью многие-ко-многим с сущностью "Цвет",
    поскольку у каждого товара может быть множество цветов, так и каждый из цветов может быть связан со множеством товаров. 
    \item <<Без связи>> – технические, узконаправленные таблицы. К ним относятся таблицы предназначенные для хранения информации, 
    необходимой для идентификации и аутентификации пользователей в системе, а также таблица, содержащая заранее определённые шаблоны почтовых сообщений с темами почтовых сообщений.
\end{enumerate}