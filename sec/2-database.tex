\subsection{Проектирование БД}\label{subsec:design-db}\indent

Спроектированная диаграмма \hyperlink{gloss:db}{БД} представлена в Приложении \ref{addition:rest-database}.

На данной диаграмме, можно выделить несколько видов связей между таблицами:

\begin{enumerate}
    \item <<один-ко-многим>> – примером такой связи может выступать связь между сущностями "Категория" и "Товар", поскольку, одновременно товар может находиться только в одной категории.
    \item <<многие-ко-многим>> – пеализуется путём применения связей 1-к-многим через промежуточную таблицу. Так, например, сущность "Товар" связана связью многие-ко-многим с сущностью "Цвет",
    поскольку у каждого товара может быть множество цветов, так и каждый из цветов может быть связан со множеством товаров. 
    \item <<без связи>> – технические, узконаправленные таблицы. К ним относятся таблицы необходимые для хранения информации необходимой для идентификации и аутентификации пользователей в системе, 
    а также таблица содержащая в себе заранее определённые шаблоны почтовых сообщений с темами сообщений.
\end{enumerate}