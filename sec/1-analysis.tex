\section[Глава 1 Анализ предметной области клиент-серверных приложений]{Глава 1 \break Анализ предметной области клиент-серверных приложений}
\label{sec:charpter-1-analisis}

\subsection{Основные сведения}\label{subsec:1-common-info}\indent

Интернет-магазин – сайт, торгующий товарами посредством сети Интернет.
Позволяет пользователям, онлайн, сформировать заказ на покупку.
Типичный интернет-магазин позволяет клиенту просматривать ассортимент продуктов и услуг фирмы, фотографии или изображения продуктов, а также информацию о технических характеристиках продуктов и ценах.
Интернет-магазины обычно позволяет покупателям использовать функции поиска.

Клиент-сервер – вычислительная или сетевая архитектура, в которой задачи распределены между поставщиками услуг (сервера), и заказчиками (клиенты).

Клиент и сервер являются программным обеспечением, которое расположено на разных вычислительных машинах и взаимодействуют друг с другом с помощью вычислительной сети, посредством сетевых протоколов.
Серверы ожидают от клиентских программ запросы и предоставляют им свои ресурсы в виде данных, или в виде сервисных функций.
Обычно, программу-сервер размещают на специально выделенном вычислительном устройстве, которое настроено особым образом т.к. сервер может выполнять запросы от многих программ-клиентов и его производительность должна быть высокой.

К достоинствам клиент-серверной архитектуры относят:

\begin{itemize}
    \item Отсутствие дублирования кода программы-сервера программами-клиентами;
    \item Снижение требований к клиентским устройствам т.к. все вычисления выполняются на стороне сервера;
    \item Все данные хранятся на сервере, который защищен гораздо лучше большей части клиентов;
    \item Возможность организации контроля полномочий, чтоб предоставлять доступ клиентам с определёнными полномочиями.
\end{itemize}

К недостаткам относят:

\begin{itemize}
    \item Поломка на стороне сервера, может обеспечить неработоспособности всей сети приложений;
    \item Поддержка работы системы требует отдельного специалиста – системного администратора;
    \item Высокая стоимость оборудования.
\end{itemize}

\subsection{Обзор подходов реализации подобных систем}\label{subsec:1-analisis} \indent

Для построения подобных систем приложений могут использоваться множество разнообразных подходов с разным списком технологий.
Так например, глобальная архитектура системы приложений может быть монолитной, модульной либо ориентированой на сервисы.

\begin{itemize}
    \item Монолитный подход является самой старой моделью проектирования \hypertarget{gloss:software}{ПО}, поскольку именно с неё и началась разработка любого \hypertarget{gloss:software}{ПО}. 
    В рамках данного подхода можно избежать сложную структуры веб-приложения, поскольку веб-сервер будет содержать в себе всю бизнес логику, необходимую для работы приложения, а база данных будет предоставлять необходимые данные серверу. 
    Однако с развитием приложения, может накопиться множество технических долгов, которые будет всё сложнее и сложнее исправить в долгосрочной перспективе;
    \item Модульная архитектура, в отличии от монолитной, подразумевает разбитие функционала приложения на отдельные модули, каждый из которых отвечает за определённую часть функционала приложения.
    Таким образом, получается приложение, состоящее из множества монолитных модулей внутри одного приложения, притом что каждый из модулей является функционально независимым от других.
    \item Сервисы представляют собой отдельные, самодостаточные модули, обладающие своей аппаратной базой (в т.ч. каждый может обладать своей базой данных);
    В результате чего, взаимодействие между сервисам происходит асинхронно, что позволяет достичь независимого масштабирования компонентов, помимо этого, такой подход позволяет использовать различные языки программирования для реализации каждого из сервисов.       
\end{itemize}

Для реализации пользовательского интерфейса могут быть использованы как простая комбинация языков разметки, стилей и скриптов, так и специализированные фреймворки для построения \hyperlink{gloss:ui}{UI}.
В том числе, задачу рендеринга пользовательского интерфеса, формирование стилей и контента, можно переложить на сервер, а можно статично задать на клиенте. 

Подробный обзор выбранных архитектурных решений, подходов реализации и технологий будет представлен во второй главе.

\subsection{Выводы по главе 1}\label{subsec:1-conclusion}\indent

В первой главе был проведён анализ предметной области.
Были выделены основные характеристики и черты клиент-серверной архитектуры, разобраны основные понятия и приведены основные достоинства и недостатки данной архитектуры.
Также, был произведён анализ существующих решений с приведением достоинств и недостатков.
