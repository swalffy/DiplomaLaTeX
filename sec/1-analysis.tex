\section[Глава 1 Анализ предметной области клиент-серверных приложений]{Глава 1 \break Анализ предметной области клиент-серверных приложений}
\label{sec:charpter-1-analisis}

\subsection{Основные сведения}\label{subsec:1-common-info}\indent

Интернет-магазин – сайт, торгующий товарами посредством сети Интернет.
Позволяет пользователям, онлайн, сформировать заказ на покупку.
Типичный интернет-магазин позволяет клиенту просматривать ассортимент продуктов и услуг фирмы, фотографии или изображения продуктов, а также информацию о технических характеристиках продуктов и ценах.
Интернет-магазины обычно позволяет покупателям использовать функции поиска.

Клиент-сервер – вычислительная или сетевая архитектура, в которой задачи распределены между поставщиками услуг (сервера), и заказчиками (клиенты).

Клиент и сервер являются программным обеспечением, которое расположено на разных вычислительных машинах и взаимодействуют друг с другом с помощью вычислительной сети, посредством сетевых протоколов.
Серверы ожидают от клиентских программ запросы и предоставляют им свои ресурсы в виде данных, или в виде сервисных функций.
Обычно, программу-сервер размещают на специально выделенном вычислительном устройстве, которое настроено особым образом т.к. сервер может выполнять запросы от многих программ-клиентов и его производительность должна быть высокой.

К достоинствам клиент-серверной архитектуры относят:

\begin{itemize}
    \item Отсутствие дублирования кода программы-сервера программами-клиентами;
    \item Снижение требований к клиентским устройствам т.к. все вычисления выполняются на стороне сервера;
    \item Все данные хранятся на сервере, который защищен гораздо лучше большей части клиентов;
    \item Возможность организации контроля полномочий, чтоб предоставлять доступ клиентам с определёнными полномочиями.
\end{itemize}

К недостаткам относят:

\begin{itemize}
    \item Поломка на стороне сервера, может обеспечить неработоспособности всей сети приложений;
    \item Поддержка работы системы требует отдельного специалиста – системного администратора;
    \item Высокая стоимость оборудования.
\end{itemize}

\subsection{Анализ существующих решений}\label{subsec:1-analisis} \indent

\TODO{провести анализ}

\subsection{Выводы по главе 1}\label{subsec:1-conclusion}\indent

В первой главе был проведён анализ предметной области.
Были выделены основные характеристики и черты клиент-серверной архитектуры, разобраны основные понятия и приведены основные достоинства и недостатки данной архитектуры.
Также, был произведён анализ существующих решений с приведением достоинств и недостатков.
