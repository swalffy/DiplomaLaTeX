\section[Глава 1 Анализ предметной области]{Глава 1 \break Анализ предметной области}
\label{sec:charpter-1-analisis}

\subsection{Основные сведения}\label{subsec:1-common-info}\indent

Интернет-магазин – сайт, торгующий товарами посредством сети Интернет.
Позволяет пользователям, сформировать заказ на покупку онлайн.
Типичный интернет-магазин позволяет клиенту просматривать ассортимент продуктов и услуг фирмы, фотографии или изображения продуктов, 
а также информацию о технических характеристиках продуктов и ценах.
Интернет-магазины обычно позволяет покупателям использовать функции поиска.

Клиент-сервер – вычислительная или сетевая архитектура, в которой задачи распределены между поставщиками услуг (сервера), и заказчиками (клиенты).

Клиент и сервер являются \hyperlink{gloss:software}{ПО}, которое расположено на разных вычислительных машинах и взаимодействуют друг с другом с помощью вычислительной сети, 
посредством сетевых протоколов.
Серверы ожидают от клиентских программ запросы и предоставляют им свои ресурсы в виде данных, или в виде сервисных функций.
Обычно, программу-сервер размещают на специально выделенном вычислительном устройстве, которое настроено особым образом, т.к. 
сервер может выполнять запросы от многих программ-клиентов и его производительность должна быть высокой \cite{client-server}.

К достоинствам клиент-серверной архитектуры относят:

\begin{itemize}
    \item отсутствие дублирования кода программы-сервера программами-клиентами;
    \item снижение требований к клиентским устройствам, т.к. все вычисления выполняются на стороне сервера;
    \item все данные хранятся на сервере, который защищен гораздо лучше большей части клиентов;
    \item возможность организации контроля полномочий, чтобы предоставлять доступ клиентам с определёнными полномочиями.
\end{itemize}

К недостаткам относят:

\begin{itemize}
    \item Поломка на стороне сервера, может привести к неработоспособности всей сети приложений;
    \item Поддержка работы системы требует отдельного специалиста – системного администратора;
    \item Высокая стоимость сетевого оборудования.
\end{itemize}

\subsection{Обзор подходов реализации клиент-серверных приложений}\label{subsec:1-analisis} \indent

Для построения подобных систем приложений могут использоваться множество разнообразных подходов с разным списком технологий.
Так например, глобальная архитектура системы приложений может быть монолитной, модульной либо ориентированой на сервисы.

\begin{itemize}
    \item Монолитный подход является наиболее старой моделью построения \hyperlink{gloss:software}{ПО}. 
    Именно с такой архитектурой начиналась разработка любого \hyperlink{gloss:software}{ПО}. 
    Используя данный подход можно избежать сложную архитектуру веб-приложения, 
    поскольку веб-сервер будет содержать в себе весь функционал необходимый для реализации бизнес логики, а база данных будет предоставлять необходимые данные серверу. 
    Однако данная архитектура не является хорошо масштабируемой и в следствии развития приложения, 
    может накопиться множество технических долгов, которые в последствии будет значительно сложнее исправлять.
    \item Модульная архитектура, подразумевает под собой разделение функционала приложения на отдельные модули, 
    каждый из которых ответственен за определённую часть функционала.
    Так, получается приложение, состоящее из большого колличества монолитных модулей внутри одного приложения и каждый из модулей получается функционально независимым от других подобных модулей.
    Однако с таким подходом реализации архитектуры, стоит учитывать, что даже на начальном этапе разработки, стоимость на порядок выше, чем при применении монолитной архитектуры.
    \item Сервисы представляют собой отдельные, самодостаточные модули, обладающие своей аппаратной базой (в т.ч. каждый может обладать своей базой данных).
    Таким образом, взаимодействие между сервисам происходит асинхронно, 
    что позволяет достичь независимого масштабирования компонентов приложения (опираясь на необходимости конкретного сервиса) 
    и позволяет использовать различные языки программирования для реализации каждого из сервисов.
    Однако, такая архитектура является наиболее дорогой в разработке и помимо этого, 
    обязывает строго определить и следовать правилам взаимодейтвия между сервисами, их \hyperlink{gloss:api}{API} и модели данных.
\end{itemize}

Для реализации пользовательского интерфейса могут быть использованы как простая комбинация языков разметки, стилей и скриптов, так и специализированные фреймворки для построения \hyperlink{gloss:ui}{UI}.
В том числе, задачу рендеринга пользовательского интерфеса, формирование стилей и контента, можно переложить на сервер, а можно статично задать на клиенте. 

Подробный обзор выбранных архитектурных решений, подходов реализации и технологий представлен во второй главе.

\subsection{Выводы по главе 1}\label{subsec:1-conclusion}\indent

В первой главе проведён анализ предметной области.
Выделены основные характеристики и черты клиент-серверной архитектуры, разобраны основные понятия и приведены основные достоинства и недостатки каждой архитектуры.
Также, произведён анализ существующих решений с приведением достоинств и недостатков.
