\addcontentsline{toc}{section}{Введение}\label{sec:intro}
\section*{Введение}\indent

В связи с развитием компьютерных технологий, жизнь человека упрощается и появляются новые решения, которые делают её лучше и удобнее. 
Раньше, для покупки необходимого товара, человеку требовалось покидать свой дом и отправляться на его поиски.
В эпоху Интернет-технологий стали появляться интернет-магазины, которые предоставили возможность гораздо быстрее находить и получать желаемые товары и продукты, оставаясь дома.

В современном мире уже сложно найти достаточно крупную организацию, которая не имеет в своём распоряжении работающий интернет-магазин.
Хорошо налаженный интернет магазин может повысить производительность организации в целом и сократить некоторые расходы.
% В современном мире существует бесчисленное множество разнообразнейших приложений. 
% Каждое из таких приложений может обладать своим выделенным сервером, однако данное решение нецелесообразно и может вызвать разное поведение приложения на разных платформах.

Одним из возможных решений может являться создание отдельного сервера, который будет возвращать единые, для всех платформ данные. 
В таком случае, любое клиентское приложение будет работать с единой базой данных и т.д., а поведение приложения будет отличаться только в случае различной реализации клиентского приложения.

Для достижения поставленной цели предусмотрены решения следующих задач:
\begin{enumerate}
    \item Анализ и формулировка требований к разрабатываемым приложениям;
    \item Обзор и выбор средств разработки;
    \item Проектирование архитектуры приложения;
    \item Проектирование \hyperlink{gloss:db}{БД};
    \item Проектирование \hyperlink{gloss:api}{API};
    \item \hyperlink{gloss:ui}{UI}/UX дизайн;
    \item Реализация приложения \hyperlink{gloss:rest}{REST-сервера};
    \item Реализация Web-клиента;
    \item Реализация Android-клиента.
\end{enumerate}