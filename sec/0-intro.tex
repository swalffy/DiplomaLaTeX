\addcontentsline{toc}{section}{Введение}\label{sec:intro}
\section*{Введение}\indent

В связи с развитием технологий, жизнь человека упрощается и появляются новые решения, которые делают её лучше и удобнее. Раньше, для покупки какого-нибудь товара, человеку было необходимо покидать свой дом и отправляться на поиски определённой вещи. В эпоху Интернет-технологий стали появляться интернет-магазины, которые предоставили возможность гораздо быстрее находить и получать желаемые товары и продукты, находясь при этом у себя дома.

В современном мире уже сложно найти достаточно крупную организацию, которая не имеет в своём распоряжении работающий интернет-магазин. Хорошо налаженный интернет магазин может повысить производительность организации в целом и сократить некоторые расходы.

В современном мире существует бесчисленное множество разнообразнейших приложений. Каждое из таких приложений может обладать своим выделенным сервером, однако данное решение нецелесообразно и может вызвать разное поведение приложения на разных платформах.

Одним из возможных решений может являться создание отдельного сервера, который будет возвращать единые, для всех платформ данные. В таком случае, между всеми клиентскими приложениями будет единая база данных и т.д., а поведение приложения будет отличаться только в случае различной реализации клиентского приложения.

Целью работы является - разработка веб-приложения для организации передачи потоковых аудио/видео данных между браузерами, с целью налаживания рабочего процесса проведения интервью.

Для достижения поставленной цели предусмотрены решения следующих задач:

\begin{enumerate}
    \item Анализ задачи и формулировка требований к разрабатываемым приложениям
    \item Обзор и выбор средств разработки
    \item Проектирование архитектуры приложения
    \item Проектирование базы данных
    \item Проектирование API
    \item UI/UX дизайн
    \item Реализация приложения REST-сервера
    \item Реализация Web-клиента
    \item Реализация Android-клиента
\end{enumerate}

Первый раздел пояснительной записки включает в себя анализ предметной области. Второй раздел посвящен проектированию системы, построению алгоритма работы программы. Третий раздел отражает реализацию программы, механизм и результаты ее работы.
