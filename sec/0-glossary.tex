\section*{Перечень условных обозначений}\label{sec:glossary}\indent

\hypertarget{gloss:api}{
    API (Application Programming Interface) – набор готовых классов, процедур, функций, структур и констант, предоставляемых приложением (библиотекой) для использования во внешних программных продуктах. 
    Часто выполняет роль слоя абстракции, упрощающего доступ к функциям приложения (библиотеки). 
    Используется для написания всевозможных приложений, основанных на готовом программном решении \cite{api}.
}

\hypertarget{gloss:software}{
    ПО (Программное обеспечение) – совокупность программ системы обработки информации и программных документов, необходимых для эксплуатации этих программ.
}

\hypertarget{gloss:rest}{
    REST API (Representational State Transfer) --– архитектурный стиль взаимодействия компонентов клиент-серверного приложения в сети.
    Такой подход помогает поддерживать несколько клиентских приложений на разных платформах, а также позволяет поддерживать достаточный уровень абстрагированности и масштабируемости.
    Представляет собой согласованный набор ограничений, учитываемых при проектировании распределённой системы.
    В определённых случаях это приводит к повышению производительности и упрощению архитектуры.
}

\hypertarget{gloss:ui}{
    UI (user interface, пользовательский интерфейс) – способ взаимодействия между пользователем и приложением.
}

\hypertarget{gloss:cms}{
    CMS (сontent management system, система управления контентом) – приложение либо их связка, которые используются для создания и управления цифровым контентом.
}

\hypertarget{gloss:jvm}{
    JVM (Java Virtual Machine) – виртуальная машина которая позволяет компьютеру запускать Java приложения ровно так же, как и программы которые были написаны на других языках, которые компилируются в Java байт-код \cite{jvm}.
}

\hypertarget{gloss:di}{
    DI (Dependency Injection) – подход использующийся в объектно-ориентированных языках, для внедрения необходимых зависимостей.
}

\hypertarget{gloss:ioc}{
    IoC (Inversion of Controle) – принцип использующийся в объектно-ориентированных языках, для повышения модульности приложения и возможности сделать его расширяемым. 
    Cуть заключается в разделении зависимостей между высокоуровневыми и низкоуровневыми слоями приложения через ряд абстракций.
}

\hypertarget{gloss:srp}{
    SRP (Single Responsibility Principle) – принцип примененение которого подразумевает, что каждый модуль или класс должен иметь лишь выполняемую задачу, 
    которая инкапсулированна в классе, модуле или функции \cite{srp}.
}

\hypertarget{gloss:jpa}{
    JPA (Java Persistence API) – Java EE/SE спецификация, описывающая систему управления сохранением Java объектов в таблицы реляционных БД в удобном виде, с помощью аннотаций \cite{jpa}.
}

\hypertarget{gloss:db}{
    БД (база данных) – организованная и структурированная коллекция данных, которая обычно хранится и доступна через компьютерную систему.
}

\hypertarget{gloss:dsl}{
    DSL (Domain Specific Language, предметно-ориентированный язык) – язык программирования разработанный для решения определённого (крайне узкого) списка задач. 
}

\hypertarget{gloss:smtp}{
    SMPT (Simple Mail Transfer Protocol, простой протокол передачи почты) – широко использующийся сетевой протокол, предназначеный для передачи почтовых e-mail сообщений в сетях TCP/IP \cite{smpt}. 
}

\hypertarget{gloss:http}{
    HTTP (Hypertext Transfer Protocol, протокол пересылки гипертекста) – протокол для пересылки гипермедиа документов, таких как HTML.
    Был разработан для сообщения между веб-браузерами и веб-серверами, однако может быть использован и для других задач \cite{http}.
}

\hypertarget{gloss:oauth}{
    OAuth – открытый протокол авторизации, который позволяет предоставить третьей стороне ограниченный доступ к защищенным ресурсам пользователя без необходимости передавать ей логин и пароль.
}

\clearpage