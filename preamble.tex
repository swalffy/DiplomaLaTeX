\usepackage{fontspec} % XeTeX
\usepackage{xunicode} % Unicode для XeTeX
\usepackage[utf8x]{inputenc}
\usepackage[english,russian]{babel}

%code listings
\usepackage{listings}

\usepackage[dvipsnames]{xcolor}
\lstdefinelanguage{Kotlin}{
    comment=[l]{//},
    commentstyle={\color{gray}\ttfamily},
    emph={delegate, filter, first, firstOrNull, forEach, lazy, map, mapNotNull, println, return@},
    emphstyle={\color{OrangeRed}},
    identifierstyle=\color{black},
    keywords={abstract, actual, as, as?, break, by, class, companion, continue, data, do, dynamic, else, enum, expect, false, final, for, fun, get, if, import, in, interface, internal, is, null, object, override, package, private, public, return, set, super, suspend, this, throw, true, try, typealias, val, var, vararg, when, where, while},
    keywordstyle={\color{NavyBlue}\bfseries},
    morecomment=[s]{/*}{*/},
    morestring=[b]",
    morestring=[s]{"""*}{*"""},
    ndkeywords={@Deprecated, @JvmField, @JvmName, @JvmOverloads, @JvmStatic, @JvmSynthetic, Array, Byte, Double, Float, Int, Integer, Iterable, Long, Runnable, Short, String},
    ndkeywordstyle={\color{BurntOrange}\bfseries},
    sensitive=true,
    stringstyle={\color{ForestGreen}\ttfamily},
}

\definecolor{lightgray}{rgb}{.93,.93,.93}
\definecolor{darkgray}{rgb}{.4,.4,.4}
\definecolor{purple}{rgb}{0.65, 0.12, 0.82}

\lstdefinelanguage{TypeScript}{
    keywords={this, typeof, new, true, false, catch, function, return, null, catch, switch, var, if, in, while, do, else, case, break},
    ndkeywords={private, class, export, boolean, number, throw, implements, import, this},
    sensitive=false,
    comment=[l]{//},
    morecomment=[s]{/*}{*/},
    morestring=[b]',
    morestring=[b]"
}

\lstdefinelanguage{Kotlin}{
    comment=[l]{//},
    emph={delegate, filter, first, firstOrNull, forEach, lazy, map, mapNotNull, println, return@},
    keywords={abstract, actual, as, as?, break, by, class, companion, continue, data, do, dynamic, else, enum, expect, false, final, for, fun, get, if, import, in, interface, internal, is, null, object, override, package, private, public, return, set, super, suspend, this, throw, true, try, typealias, val, var, vararg, when, where, while},
    morecomment=[s]{/*}{*/},
    morestring=[b]",
    morestring=[s]{"""*}{*"""},
    ndkeywords={@Deprecated, @JvmField, @JvmName, @JvmOverloads, @JvmStatic, @JvmSynthetic, Array, Byte, Double, Float, Int, Integer, Iterable, Long, Runnable, Short, String},
    sensitive=true,
}

\lstset{
    extendedchars=true,
    basicstyle=\footnotesize\ttfamily,
    showstringspaces=false,
    showspaces=false,
    numbers=left,
    numberstyle=\footnotesize,
    numbersep=8pt,
    tabsize=2,
    breaklines=true,
    showtabs=true,
    captionpos=b,
    identifierstyle=\color{black},
    emphstyle={\color{OrangeRed}},
    backgroundcolor=\color{lightgray},
    keywordstyle={\color{NavyBlue}\bfseries},
    commentstyle=\color{purple}\ttfamily,
    ndkeywordstyle={\color{Blue}\bfseries},
    stringstyle={\color{ForestGreen}\ttfamily},
}


% Шрифты, xelatex
\defaultfontfeatures{Ligatures=TeX}
\setmainfont{Times New Roman} % Нормоконтроллеры хотят именно его
\newfontfamily\cyrillicfont{Times New Roman}
% \setsansfont{Liberation Sans} % Тут я его не использую, но если пригодится
\setmonofont{FreeMono} % Моноширинный шрифт для оформления кода

% Русский язык
\usepackage{polyglossia}
\setdefaultlanguage{russian}

\usepackage{enumerate}
\usepackage{indentfirst}
\usepackage{float}

%images
\usepackage{graphicx}
\graphicspath{{images/}}
\usepackage{chngcntr}

\usepackage{hyperref}
\hypersetup{
    colorlinks, urlcolor={black}, % Все ссылки черного цвета, кликабельные
    linkcolor={black}, citecolor={black}, filecolor={black}
}

\renewcommand{\baselinestretch}{1.5} % Полуторный межстрочный интервал
\parindent 1.63cm % Абзацный отступ

\sloppy             % Избавляемся от переполнений
\hyphenpenalty=1000 % Частота переносов
\clubpenalty=10000  % Запрещаем разрыв страницы после первой строки абзаца
\widowpenalty=10000 % Запрещаем разрыв страницы после последней строки абзаца

%page paddings
\usepackage{geometry}
\geometry{left=3cm}
\geometry{right=1cm}
\geometry{top=2cm}
\geometry{bottom=2cm}

\usepackage{enumitem}
\setlist[enumerate,itemize]{leftmargin=15mm} % Отступы в списках

\makeatletter
\AddEnumerateCounter{\asbuk}{\@asbuk}{м)}
\makeatother
\setlist{nolistsep}                           % Нет отступов между пунктами списка
\renewcommand{\labelitemi}{\bfseries\textbullet} % Маркер списка
\renewcommand{\labelenumii}{\theenumii}
\renewcommand{\theenumii}{\theenumi.\arabic{enumii}.}

% Содержание
\usepackage{tocloft}
\renewcommand{\cfttoctitlefont}{\hspace{0.38\textwidth}\MakeTextUppercase} % СОДЕРЖАНИЕ
\renewcommand{\cftsecfont}{\hspace{0pt}}            % Имена секций в содержании не жирным шрифтом
\renewcommand\cftsecleader{\cftdotfill{\cftdotsep}} % Точки для секций в содержании
\renewcommand\cftsecpagefont{\mdseries}             % Номера страниц не жирные
\setcounter{tocdepth}{2}                            % Глубина оглавления, до subsubsection

% Нумерация страниц посередине сверху
\usepackage{fancyhdr}
\pagestyle{fancy}
\fancyhf{}
\cfoot{\textrm{\thepage}}
\fancyheadoffset{0mm}
\fancyfootoffset{0mm}
\setlength{\headheight}{17pt}
\renewcommand{\headrulewidth}{0pt}
\renewcommand{\footrulewidth}{0pt}
\fancypagestyle{plain}{
    \fancyhf{}
    \cfoot{\textrm{\thepage}}
}

% Формат подрисуночных надписей
\RequirePackage{caption}
\DeclareCaptionLabelSeparator{defffis}{ -- } % Разделитель
\captionsetup[figure]{justification=centering, labelsep=defffis, format=plain, labelfont=bf, textfont=bf} % Подпись рисунка по центру
\captionsetup[table]{justification=raggedright, labelsep=defffis, format=plain, singlelinecheck=false} % Подпись таблицы слева
\captionsetup[lstlisting]{justification=centering, labelsep=defffis, format=plain, labelfont=bf, textfont=bf} % Подпись листинга по центру
\addto\captionsrussian{\renewcommand{\figurename}{Рисунок}} % Имя фигуры
\addto\captionsrussian{\renewcommand{\lstlistingname}{Листинг}} % TODO check listing numbering
%\addto\captionsrussian{\renewcommand{\lstlistingname}{Листинг \thesection.\arabic{lstlisting}}}
\renewcommand{\thefigure}{\thesection.\arabic{figure}}

% Заголовки секций в оглавлении в верхнем регистре
\usepackage{textcase}
\makeatletter
\let\oldcontentsline\contentsline
\def\contentsline#1#2{
    \expandafter\ifx\csname l@#1\endcsname\l@section
    \expandafter\@firstoftwo
    \else
    \expandafter\@secondoftwo
    \fi
    {\oldcontentsline{#1}{\MakeTextUppercase{#2}}}
    {\oldcontentsline{#1}{#2}}
}
\makeatother

% Оформление заголовков
\usepackage[compact,explicit]{titlesec}
\titleformat{\section}{\Large\bfseries}{}{0mm}{\clearpage\centering{\MakeUppercase{#1}}\vspace{1.5em}\setcounter{figure}{0}\setcounter{lstlisting}{0}}
\titleformat{\subsection}[block]{\large\bfseries}{}{1.5cm}{\thesubsection\quad#1}
\titleformat{\subsubsection}[block]{\normalsize\bfseries}{}{15mm}{\thesubsubsection\quad#1}
\titleformat{\paragraph}[block]{\normalsize}{}{12.5mm}{\MakeTextUppercase{#1}}
\titlespacing*{\subsection}{0mm}{0.3em}{0.2em}

\usepackage{lastpage} % Подсчет количества страниц
\setcounter{page}{2}  % Начало нумерации страниц

% Пользовательские функции
% Секции без номеров (введение, заключение...), вместо section*{}
\newcommand{\anonsection}[1]{
    \section{#1}\label{sec:#1}
    \addcontentsline{toc}{section}{#1}
}

\newcommand{\addimg}[4]{ % Добавление одного рисунка
    \begin{figure}
        \centering
        \includegraphics[width=#2\linewidth]{#1}
        \caption{\bfseries{#3}} \label{#1}
    \end{figure}
}
\newcommand{\addimghere}[4]{ % Добавить рисунок непосредственно в это место
    \begin{figure}[H]
        \centering
        \includegraphics[width=#2\linewidth]{#1}
        \caption{#3} \label{#1}
    \end{figure}
}

\definecolor{shadecolor}{RGB}{255,255,0}
\newcommand{\TODO}[1]{\setlength\fboxsep{15pt}\par\noindent\colorbox{shadecolor}
{\textcolor{red}{\bfseries\parbox{\dimexpr\textwidth-2\fboxsep\relax}{TODO: #1}}}}

\usepackage{xassoccnt}
\NewTotalDocumentCounter{totalfigures}
\NewTotalDocumentCounter{totallistings}
\DeclareAssociatedCounters{figure}{totalfigures}
\DeclareAssociatedCounters{lstlisting}{totallistings}

\usepackage[autostyle]{csquotes}