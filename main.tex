% !TEX TS-program = xelatex
\documentclass[14pt,a4paper]{scrartcl}

\usepackage{fontspec} % XeTeX
\usepackage{xunicode} % Unicode для XeTeX
\usepackage[utf8x]{inputenc}
\usepackage[english,russian]{babel}

%code listings
\usepackage{listings}

\usepackage[dvipsnames]{xcolor}

\definecolor{lightgray}{rgb}{.93,.93,.93}
\definecolor{darkgray}{rgb}{.4,.4,.4}
\definecolor{purple}{rgb}{0.65, 0.12, 0.82}

\lstdefinelanguage{TypeScript}{
  keywords={this, typeof, new, true, false, catch, function, return, null, catch, switch, var, if, in, while, do, else, case, break},
  ndkeywords={private, class, export, boolean, number, throw, implements, import, this},
  sensitive=false,
  comment=[l]{//},
  morecomment=[s]{/*}{*/},
  morestring=[b]',
  morestring=[b]"
}

\lstdefinelanguage{Kotlin}{
  comment=[l]{//},
  keywords={init, inner, abstract, actual, as, as?, break, by, class, companion, continue, data, do, dynamic, else, enum, expect, false, final, for, fun, get, if, import, in, interface, internal, is, null, object, override, package, private, public, return, set, super, suspend, this, throw, true, try, typealias, val, var, vararg, when, where, while},
  morecomment=[s]{/*}{*/},
  morestring=[b]",
  morestring=[s]{"""*}{*"""},
  ndkeywords={@Deprecated, @JvmField, @JvmName, @JvmOverloads, @JvmStatic, @JvmSynthetic, Array, Byte, Double, Float, Int, Integer, Iterable, Long, Runnable, Short, String, Unit},
  sensitive=true,
}

\lstdefinelanguage{HTML5}{
  language=html,
  sensitive=true,
  alsoletter={<>=-},
  otherkeywords={
    <html>, <head>, <title>, </title>, <meta, />, </head>, <body>,
    <form, </form>,
    <mat-horizontal-stepper, </mat-horizontal-stepper>,
    <mat-step, </mat-step>,
    <mat-form-field, <\/mat-form-field>,
    <mat-step, </mat-step>,
    <mat-checkbox, </mat-checkbox>,
    <ng-container, </ng-container>,
    <mat-slide-toggle, </mat-slide-toggle>,
    <mat-card>, </mat-card>,
    <mat-card-title>, </mat-card-title>,
    <mat-card-subtitle>, </mat-card-subtitle>,
    <mat-card-content>, </mat-card-content>,
    <mat-divider>,</mat-divider>,
    <mat-list, </mat-list>,
    <mat-list-item, </mat-list-item>,
    <table, <tr>, <td>, </tr>, </td>, </table>,
    <input, <mat-error, </mat-error>,
    <div>, </div>, <div ,
    <span, </span>, <button, </button>,
    <canvas, \/canvas>, <script>, </script>, </body>, </html>, <!, html>, <style>, </style>, ><
  },
  ndkeywords={
    =, matInput, ngIf, ngFor, ngSwitchCase, mat-raised-button, matStepperNext, matStepperPrevious,
    charset=, id=, width=, height=,
    border:, transform:, -moz-transform:, transition-duration:, transition-property:, transition-timing-function:
  },
  morecomment=[s]{<!--}{-->},
  tag=[s]
}

\lstdefinelanguage{CSS}{
  morestring=[s]{:}{;},
  sensitive,
  morecomment=[s]{/*}{*/}
}

\lstset{
  extendedchars=true,
  basicstyle=\linespread{0.3}\ttfamily\scriptsize,
  showstringspaces=false,
  showspaces=false,
  numbers=left,
  numberstyle=\footnotesize,
  numbersep=8pt,
  tabsize=2,
  breaklines=true,
  showtabs=true,
  captionpos=b,
  identifierstyle=\color{black},
  emphstyle={\color{OrangeRed}},
  keywordstyle={\color{NavyBlue}\bfseries},
  commentstyle=\color{purple}\ttfamily,
  ndkeywordstyle={\color{Blue}\bfseries},
  stringstyle={\color{ForestGreen}\ttfamily},
}

% Шрифты, xelatex
\defaultfontfeatures{Ligatures=TeX}
\setmainfont{Times New Roman}
\newfontfamily\cyrillicfont{Times New Roman}
\setmonofont{FreeMono} % Моноширинный шрифт для оформления кода

% Русский язык
\usepackage{polyglossia}
\setdefaultlanguage{russian}

\usepackage{enumerate}
\usepackage{indentfirst}
\usepackage{float}

%images
\usepackage{graphicx}
\graphicspath{{images/}}
\usepackage{chngcntr}

\usepackage{hyperref}
\hypersetup{
  colorlinks, urlcolor={black}, % Все ссылки черного цвета, кликабельные
  linkcolor={black}, citecolor={black}, filecolor={black}
}
\urlstyle{rm}

\renewcommand{\baselinestretch}{1.5} % Полуторный межстрочный интервал
\parindent 12mm % Абзацный отступ

\sloppy             % Избавляемся от переполнений
\hyphenpenalty=10000 % Частота переносов
% \widowpenalties=3 10000 10000 150
\clubpenalty=10000  % Запрещаем разрыв страницы после первой строки абзаца
% \widowpenalty=10000 % Запрещаем разрыв страницы после последней строки абзаца
% 
%page paddings
\usepackage[
  left=3cm,
  right=1cm,
  top=2cm,
  bottom=2cm
]{geometry}

\usepackage{enumitem}
\setlist[enumerate,itemize]{leftmargin=15mm, nolistsep} % Отступы в списках
% \setitemize[1]{labelindent=4cm,itemindent=4cm}

\makeatletter
\AddEnumerateCounter{\asbuk}{\@asbuk}{м)}
\makeatother
\renewcommand{\labelitemi}{\bfseries\textbullet} % Маркер списка
\renewcommand{\labelenumii}{\theenumii}
\renewcommand{\theenumii}{\theenumi.\arabic{enumii}.}

% Содержание
\usepackage{tocloft}
\renewcommand{\cfttoctitlefont}{\hspace{0.38\textwidth}\MakeTextUppercase} % СОДЕРЖАНИЕ
\renewcommand{\cftsecfont}{\hspace{0pt}}            % Имена секций в содержании не жирным шрифтом
\renewcommand\cftsecleader{\cftdotfill{\cftdotsep}} % Точки для секций в содержании
\renewcommand\cftsecpagefont{\mdseries}             % Номера страниц не жирные
\setcounter{tocdepth}{2}                            % Глубина оглавления, до subsubsection
\addtocontents{toc}{\protect\thispagestyle{empty}}

% Нумерация страниц посередине сверху
\usepackage{fancyhdr}
\pagestyle{fancy}
\fancyhf{}
\rhead{\textrm{\thepage}}
\renewcommand{\headrulewidth}{0pt}
\renewcommand{\footrulewidth}{0pt}
\setlength{\headheight}{0pt}
\setlength{\footheight}{0pt}
\fancypagestyle{plain}{
  \fancyhf{}
  \rhead{\textrm{\thepage}}
}

% Формат подрисуночных надписей
\RequirePackage{caption}
\DeclareCaptionLabelSeparator{defffis}{ -- } % Разделитель
\captionsetup[figure]{font=small, justification=centering, labelsep=defffis, format=plain, labelfont=bf, textfont=bf} % Подпись рисунка по центру
\captionsetup[table]{font=small, justification=raggedright, labelsep=defffis, format=plain, singlelinecheck=false} % Подпись таблицы слева
\captionsetup[lstlisting]{justification=centering, labelsep=defffis, format=plain, labelfont=bf, textfont=bf} % Подпись листинга по центру
\addto\captionsrussian{\renewcommand{\figurename}{Рисунок}} % Имя фигуры
\addto\captionsrussian{\renewcommand{\lstlistingname}{Листинг}} % TODO check listing numbering
%\addto\captionsrussian{\renewcommand{\lstlistingname}{Листинг \thesection.\arabic{lstlisting}}}
\renewcommand{\thefigure}{\thesection.\arabic{figure}}

% Заголовки секций в оглавлении в верхнем регистре
\usepackage{textcase}
\makeatletter
\let\oldcontentsline\contentsline
\def\contentsline#1#2{
  \expandafter\ifx\csname l@#1\endcsname\l@section
  \expandafter\@firstoftwo
  \else
  \expandafter\@secondoftwo
  \fi
  {\oldcontentsline{#1}{\MakeTextUppercase{#2}}}
  {\oldcontentsline{#1}{#2}}
}
\makeatother

% Оформление заголовков
\usepackage[compact,explicit]{titlesec}
\titleformat{\section}{\large\bfseries}{}{0mm}{\clearpage\centering{\MakeUppercase{#1}}\vspace{18pt}\setcounter{figure}{0}\setcounter{lstlisting}{0}}
\titleformat{\subsection}[block]{\fontsize{15}{18}\selectfont\bfseries}{}{1.5cm}{\thesubsection\quad#1}
\titleformat{\subsubsection}[block]{\normalsize\bfseries}{}{15mm}{\thesubsubsection\quad#1}
\titleformat{\paragraph}[block]{\normalsize}{}{12.5mm}{\MakeTextUppercase{#1}}
\titlespacing*{\subsection}{0mm}{0.3em}{0.2em}

\usepackage{lastpage} % Подсчет количества страниц
\setcounter{page}{2}  % Начало нумерации страниц

% Пользовательские функции
% Секции без номеров (введение, заключение...), вместо section*{}
\newcommand{\anonsection}[1]{
  \section*{#1}\label{sec:#1}
  \addcontentsline{toc}{section}{#1}
}

\newcommand{\addimg}[4]{ % Добавление одного рисунка
  \begin{figure}
    \centering
    \includegraphics[width=#2\linewidth]{#1}
    \caption{\small\bfseries{#3}} \label{#1}
  \end{figure}
}
\newcommand{\addimghere}[4]{ % Добавить рисунок непосредственно в это место
  \begin{figure}[H]
    \centering
    \includegraphics[width=#2\linewidth]{#1}
    \ifx&#3&%
    \else
    \caption{\small#3} \label{#1}
    \fi
  \end{figure}
}

\newcommand\capmystring[1]{\capmystringaux#1\relax}
\def\capmystringaux#1#2\relax{\uppercase{#1}\lowercase{#2}}
%\newcounter{additionCounter}
\newcommand{\addition}[3] {
  \clearpage
  \refstepcounter{additionCounter}
  \label{addition:#1}
  \addcontentsline{toc}{section}{
    \emph{\capmystring{Приложение} \theadditionCounter}. \capmystring{#2}
  }

  \begin{flushright}
    \MakeUppercase{Приложение} \theadditionCounter
  \end{flushright}

  \begin{center}
    \large\bfseries#2
  \end{center}

  #3
}

\definecolor{shadecolor}{RGB}{255,255,0}
\newcommand{\TODO}[1]{\setlength\fboxsep{15pt}\par\noindent\colorbox{shadecolor}
{\textcolor{red}{\bfseries\parbox{\dimexpr\textwidth-2\fboxsep\relax}{TODO: #1}}}}

% Declare counters

\usepackage{xassoccnt}
\NewTotalDocumentCounter{totalfigures}
\NewTotalDocumentCounter{totallistings}
\NewTotalDocumentCounter{additionCounter}
\NewTotalDocumentCounter{biblioCounter}
\DeclareAssociatedCounters{figure}{totalfigures}
\DeclareAssociatedCounters{lstlisting}{totallistings}
\setcounter{additionCounter}{0}

\def\oldbibitem{} \let\oldbibitem=\bibitem
\def\bibitem{\stepcounter{biblioCounter}\oldbibitem}

\usepackage[autostyle]{csquotes}

\begin{document}

    \pagestyle{empty}

    \begin{titlepage}
    \begin{center}
        МИНИСТЕРСТВО ОБРАЗОВАНИЯ РЕСПУБЛИКИ БЕЛАРУСЬ

        \vspace{0.5cm}

        Учреждение образования \\
        «Гродненский государственный университет имени Янки Купалы»
        \vspace{0.5cm}

        Факультет математики и иформатики \\
        Кафедра современных технологий программирования

        \vfill

        ВАСИЛЬКОВ ВЛАДИМИР ЮРЬЕВИЧ
        \vfill

        {\large\textbf{
            Разработка комплекса приложений "Оптовая торговля"
        }}

        \vfill

        Дипломная работа \\
        студента 4 курса специальности\\
        1-40 01 01 «Программное обеспечение информационных технологий»\\
        дневной формы получения образования

        \bigskip
    \end{center}
    \vfill

    \newlength{\ML}
    \settowidth{\ML}{«\underline{\hspace{0.7cm}}» \underline{\hspace{2cm}}}
    \begin{figure}[!htb]
        \centering
        \begin{minipage}{.5\textwidth}
            «Допустить к защите»\\
            Заведующий кафедрой\\
            \underline{\hspace{\ML}} Рудикова Л.\,В.\\
            «\underline{\hspace{0.7cm}}» \underline{\hspace{2cm}} 2020г.
        \end{minipage}%
        \begin{minipage}[c]{0.5\textwidth}
            \flushright{Научный руководитель\\
            Гуща Юлия Вальдемаровна\\
            старший преподаватель кафедры современных технилогий программирования}
        \end{minipage}%
    \end{figure}

    \vfill

    \begin{center}
        Гродно 2020
    \end{center}
\end{titlepage}

    \pagestyle{empty}
\section*{Резюме}\label{sec:cw-ru}\indent

Васильков Владимир Юрьевич

Тема дипломной работы: Разработка комплекса приложений "Оптовый интернет-магазин”
\TODO{check it}

Работа содержит: \pageref{LastPage} страниц, \TotalValue{totalfigures} рисунков, \TotalValue{additionCounter} приложений, \TotalValue{totallistings} листингов, \TotalValue{biblioCounter} использованных источника литературы.

Ключевые слова: сервер, интернет-магазин, приложение, язык программирования Kotlin, \hyperlink{gloss:rest}{Rest-архитектура}, контроль доступа, клиент-серверная архитектура, Spring Boot, json, web-клиент, язык программирования Typescript, Angular, HTML, CSS, Android-клиент, Android, MVVM, Clean Architecture, Android Architecture Components.

Цель дипломной работы – изучение методов построения клиент-серверных систем приложения взаимодействующих посредством \hyperlink{gloss:rest}{rest-принципа}.

Задача дипломной работы – реализовать серверное приложение, способное взаимодействовать с разработанными клиентскими приложениями, использующие разные платформы, управлять данными, которые содержатся в базе данных, контролировать доступ и привилегии пользователей.

Объектом исследования выступают \hyperlink{gloss:rest}{rest-приложения}, построенные по принципу клиент-сервер.

Предметами исследования - основные функции и принципы функционирования клиент-серверных приложения использующих \hyperlink{gloss:rest}{rest-подход} для взаимодействия клиентом и сервером.

\section*{Summary}\label{sec:cw-eng}\indent
\TODO{Check translate}

Uladzimir Vasilkou Yurievich

Theme of diploma is: "Development of set of applications "Wholesale online store”

Diploma contains: \pageref{LastPage} pages, \TotalValue{totalfigures} images, \TotalValue{totaladditions}, \TotalValue{totallistings} listings, \TotalValue{biblioCounter} bibliography sources.

Keywords: server, online store, application, Kotlin programming language, \hyperlink{gloss:rest}{Rest-architecture}, access control, client-server architecture, Spring Boot, json, Web-client, Web-application, TypeScript programming language, Angular, HTML, CSS, Android-client, Android, MVVM, MVC, Clean Architecture, Android Architecture Components.

The purpose of the diploma is investigation and research methods of implementation client-server application systems that interact with the help of \hyperlink{gloss:rest}{rest-principle}.

The aim of the diploma to implement server app, that can interact with implemented client applications, that uses different platforms, control data in database, control access and users privileges.

The object of reserch is \hyperlink{gloss:rest}{rest-application}, that interact in client-server principle.

The research subject is the main function and principles functioning client-server applications that uses \hyperlink{gloss:rest}{rest} way to ineruct between client and server.

    \clearpage
    \tableofcontents\thispagestyle{empty}

    \anonsection{Перечень условных обозначений}\indent

\hypertarget{gloss:api}{
    API (Application Programming Interface) --- набор готовых классов, процедур, функций, структур и констант, предоставляемых приложением (библиотекой) для использования во внешних программных продуктах. 
    Часто выполняет роль слоя абстракции, упрощающего доступ к функциям приложения (библиотеки). 
    Используется для написания всевозможных приложений, основанных на готовом программном решении.\cite{api}
}

\hypertarget{gloss:software}{
    ПО (Программное обеспечение) --- совокупность программ системы обработки информации и программных документов, необходимых для эксплуатации этих программ.
}

\hypertarget{gloss:rest}{
    REST API (Representational State Transfer) --– архитектурный стиль взаимодействия компонентов клиент-серверного приложения в сети.
    Такой подход помогает поддерживать несколько клиентских приложений на разных платформах, а также позволяет поддерживать достаточный уровень абстрагированности и масштабируемости.
    Представляет собой согласованный набор ограничений, учитываемых при проектировании распределённой системы.
    В определённых случаях это приводит к повышению производительности и упрощению архитектуры.
}

\hypertarget{gloss:ui}{
    UI (User interface, Пользовательский интерфейс) --- средство взаимодействия между пользователем и компьютером.
}

\hypertarget{gloss:cms}{
    CMS (Content Management System, Система управления контентом) --- приложение либо их связка, которые используются для создания и управления цифровым контентом.
}

\hypertarget{gloss:jvm}{
    JVM (Java Virtual Machine) --- виртуальная машина которая позволяет компьютеру запускать Java приложения ровно так же, как и программы которые были написаны на других языках, которые компилируются в Java байт-код.\cite{jvm}
}

\hypertarget{gloss:di}{
    DI (Dependency Injection) --- подход использующийся в объектно-ориентированных языках, для внедрения необходимых зависимостей.
}

\hypertarget{gloss:ioc}{
    IoC (Inversion of controle) --- принцип использующийся в объектно-ориентированных языках, для повышения уровня модульности приложения и возможности сделать приложения расширяемым. 
    Его суть заключается в разделении зависимостей между высокоуровневыми и низкоуровневыми слоями приложения через ряд абстракций.
}

\hypertarget{gloss:srp}{
    SRP (Single responsibility principle) --- принцип заявляющий о том, что каждый модуль или класс должен иметь одну и только одну выполняемую задачу, которая должна быть инкапсулированна в классе, модуле или функции.\cite{srp}
}

\hypertarget{gloss:jpa}{
    JPA (Java Persistence API) --- Java EE/SE спецификация, описывающая систему управления сохранением Java объектов в таблицы реляционных БД в удобном виде, с помощью аннотаций.\cite{jpa}
}

\hypertarget{gloss:db}{
    БД (База данных) --- организованная и структурированная коллекция данных, которая обычно хранится и доступна через компьютерную систему.
}

\hypertarget{gloss:dsl}{
    DSL (Domain Specific Language, предметно-ориентированный язык) --- язык программирования разработанный для решения определённого (крайне узкого) списка задач. 
}

\hypertarget{gloss:smtp}{
    SMPT (Simple Mail Transfer Protocol, простой протокол передачи почты) --- широко использующийся сетевой протокол, предназначеный для передачи почтовых e-mail сообщений в сетях TCP/IP.\cite{smpt} 
}

\hypertarget{gloss:http}{
    HTTP (Hypertext Transfer Protocol, протокол пересылки гипертекста) --- протокол для пересылки гипермедиа документов, таких как HTML.
    Был разработан для сообщения между Web-браузерами и Web-серверами, однако может быть использован и для других задач.\cite{http}
}

\clearpage
    \pagestyle{plain}
    \addcontentsline{toc}{section}{Введение}\label{sec:intro}
\section*{Введение}\indent

В связи с развитием технологий, жизнь человека упрощается и появляются новые решения, которые делают её лучше и удобнее. Раньше, для покупки какого-нибудь товара, человеку было необходимо покидать свой дом и отправляться на поиски определённой вещи. В эпоху Интернет-технологий стали появляться интернет-магазины, которые предоставили возможность гораздо быстрее находить и получать желаемые товары и продукты, находясь при этом у себя дома.

В современном мире уже сложно найти достаточно крупную организацию, которая не имеет в своём распоряжении работающий интернет-магазин. Хорошо налаженный интернет магазин может повысить производительность организации в целом и сократить некоторые расходы.

В современном мире существует бесчисленное множество разнообразнейших приложений. Каждое из таких приложений может обладать своим выделенным сервером, однако данное решение нецелесообразно и может вызвать разное поведение приложения на разных платформах.

Одним из возможных решений может являться создание отдельного сервера, который будет возвращать единые, для всех платформ данные. В таком случае, между всеми клиентскими приложениями будет единая база данных и т.д., а поведение приложения будет отличаться только в случае различной реализации клиентского приложения.

Целью работы является - разработка веб-приложения для организации передачи потоковых аудио/видео данных между браузерами, с целью налаживания рабочего процесса проведения интервью.

Для достижения поставленной цели предусмотрены решения следующих задач:

\begin{enumerate}
    \item Анализ задачи и формулировка требований к разрабатываемым приложениям
    \item Обзор и выбор средств разработки
    \item Проектирование архитектуры приложения
    \item Проектирование базы данных
    \item Проектирование API
    \item UI/UX дизайн
    \item Реализация приложения REST-сервера
    \item Реализация Web-клиента
    \item Реализация Android-клиента
\end{enumerate}

Первый раздел пояснительной записки включает в себя анализ предметной области. Второй раздел посвящен проектированию системы, построению алгоритма работы программы. Третий раздел отражает реализацию программы, механизм и результаты ее работы.

    \section[Глава 1 Анализ предметной области клиент-серверных приложений]{Глава 1 \break Анализ предметной области клиент-серверных приложений}
\label{sec:charpter-1-analisis}

\subsection{Основные сведения}\label{subsec:1-common-info}\indent

Интернет-магазин – сайт, торгующий товарами посредством сети Интернет.
Позволяет пользователям, онлайн, сформировать заказ на покупку.
Типичный интернет-магазин позволяет клиенту просматривать ассортимент продуктов и услуг фирмы, фотографии или изображения продуктов, а также информацию о технических характеристиках продуктов и ценах.
Интернет-магазины обычно позволяет покупателям использовать функции поиска.

Клиент-сервер – вычислительная или сетевая архитектура, в которой задачи распределены между поставщиками услуг (сервера), и заказчиками (клиенты).

Клиент и сервер являются программным обеспечением, которое расположено на разных вычислительных машинах и взаимодействуют друг с другом с помощью вычислительной сети, посредством сетевых протоколов.
Серверы ожидают от клиентских программ запросы и предоставляют им свои ресурсы в виде данных, или в виде сервисных функций.
Обычно, программу-сервер размещают на специально выделенном вычислительном устройстве, которое настроено особым образом т.к. сервер может выполнять запросы от многих программ-клиентов и его производительность должна быть высокой.

К достоинствам клиент-серверной архитектуры относят:

\begin{itemize}
    \item Отсутствие дублирования кода программы-сервера программами-клиентами;
    \item Снижение требований к клиентским устройствам т.к. все вычисления выполняются на стороне сервера;
    \item Все данные хранятся на сервере, который защищен гораздо лучше большей части клиентов;
    \item Возможность организации контроля полномочий, чтоб предоставлять доступ клиентам с определёнными полномочиями.
\end{itemize}

К недостаткам относят:

\begin{itemize}
    \item Поломка на стороне сервера, может обеспечить неработоспособности всей сети приложений;
    \item Поддержка работы системы требует отдельного специалиста – системного администратора;
    \item Высокая стоимость оборудования.
\end{itemize}

\subsection{Обзор подходов реализации подобных систем}\label{subsec:1-analisis} \indent

Для построения подобных систем приложений могут использоваться множество разнообразных подходов с разным списком технологий.
Так например, глобальная архитектура системы приложений может быть монолитной, модульной либо ориентированой на сервисы.

\begin{itemize}
    \item Монолитный подход является самой старой моделью проектирования \hypertarget{gloss:software}{ПО}, поскольку именно с неё и началась разработка любого \hypertarget{gloss:software}{ПО}. 
    В рамках данного подхода можно избежать сложную структуры веб-приложения, поскольку веб-сервер будет содержать в себе всю бизнес логику, необходимую для работы приложения, а база данных будет предоставлять необходимые данные серверу. 
    Однако с развитием приложения, может накопиться множество технических долгов, которые будет всё сложнее и сложнее исправить в долгосрочной перспективе;
    \item Модульная архитектура, в отличии от монолитной, подразумевает разбитие функционала приложения на отдельные модули, каждый из которых отвечает за определённую часть функционала приложения.
    Таким образом, получается приложение, состоящее из множества монолитных модулей внутри одного приложения, притом что каждый из модулей является функционально независимым от других.
    \item Сервисы представляют собой отдельные, самодостаточные модули, обладающие своей аппаратной базой (в т.ч. каждый может обладать своей базой данных);
    В результате чего, взаимодействие между сервисам происходит асинхронно, что позволяет достичь независимого масштабирования компонентов, помимо этого, такой подход позволяет использовать различные языки программирования для реализации каждого из сервисов.       
\end{itemize}

Для реализации пользовательского интерфейса могут быть использованы как простая комбинация языков разметки, стилей и скриптов, так и специализированные фреймворки для построения \hyperlink{gloss:ui}{UI}.
В том числе, задачу рендеринга пользовательского интерфеса, формирование стилей и контента, можно переложить на сервер, а можно статично задать на клиенте. 

Подробный обзор выбранных архитектурных решений, подходов реализации и технологий будет представлен во второй главе.

\subsection{Выводы по главе 1}\label{subsec:1-conclusion}\indent

В первой главе был проведён анализ предметной области.
Были выделены основные характеристики и черты клиент-серверной архитектуры, разобраны основные понятия и приведены основные достоинства и недостатки данной архитектуры.
Также, был произведён анализ существующих решений с приведением достоинств и недостатков.

    
    % #### CHARPTER 2 - Designing ####
    \section[Глава 2 Проектирование системы разрабатываемых приложений]{Глава 2\break Проектирование системы разрабатываемых приложений}
    \label{sec:charpter-2-designing}

    \subsection{Этапы разработки системы приложений}\label{subsec:2-dev-stages}\indent

Для разработки системы приложения, необходимо разбить данный процесс на этапы и поставить ряд задач для каждого из этапов.

Работу над разработкой системы приложений можно разбить на следующие этапы:

\begin{enumerate}
    \item Постановка задачи и анализ требований
    \item Проектирование общей архитектуры системы приложений
    \begin{enumerate}
        \item Проектирование \hyperlink{gloss:rest}{Rest-сервера}
        \begin{itemize}
            \item Описание общей архитектуры приложения;
            \item Проектирование диаграммы использования;
            \item Проектирование схемы \hyperlink{gloss:db}{БД};
            \item Описание технологий, используемых в разработке;
            \item Описание основных слоёв приложения;
            \item Диаграмма классов приложения.
        \end{itemize}
        \item Проектирование Web-клиента
        \begin{itemize}
            \item Описание общей архитектуры приложения;
            \item Проектирование шаблонов основных экранов приложения;
            \item Описание технологий, используемых в разработке;
            \item Описание основных компонентов приложения;
            \item Диаграмма деятельности некоторых процессов в приложении.
        \end{itemize}
        \item Проектирование Android-клиента
        \begin{itemize}
            \item Описание общей архитектуры приложения
            \item Проектирование шаблонов основных экранов приложения
            \item Описание технологий, используемых в разработке
            \item Описание основных компонентов приложения
            \item Диаграмма деятельности некоторых процессов в приложении
        \end{itemize}
    \end{enumerate}
    \item Реализация системы приложений
    \begin{enumerate}
        \item Реализация основных модулей приложений
        \item Ad hook тестирование разработанной системы приложений как отдельно, так и их взаимодействие между собой
    \end{enumerate}
\end{enumerate}
    \subsection{Определение общего функционала приложений}\label{subsec:2-define-functionality}\indent

Исходя из выводов, сделанных в конце первой главы, мы должны определить набор функций, которые будут реализованы в системе приложений.
Поскольку будет разработано не одно приложение, а целая система, то и функционал будет разделён по принадлежности к определённому приложению.

\subsubsection{Общий функционал Rest-сервера}\indent

Разрабатываемое приложение должно реализовывать базовые функции интернет-магазина.
А именно:

\begin{itemize}
    \item Добавление/изменение/удаление продуктов из базы данных;
    \item Возможность назначение скидок определённым пользователям;
    \item Обеспечение механизмов аутентификации;
    \item Возможность формирования заказов на основе товаров, которые клиент положил в свою корзину;
    \item Рассылка e-mail сообщений на основе загруженных шаблонов e-mail сообщений для обеспечения информирование клиентов и менеджеров о состояниях заказов или об объявлениях и акциях;
    \item Управление внутренними файловыми ресурсами приложения;
    \item Реализация сервера изображений, используемых в клиентских приложениях.
\end{itemize}

\subsubsection{Общий функционал Web-клиента}\indent

Разрабатываемое приложение должно состоять из двух модулей:

\begin{enumerate}
    \item Пользовательская часть;
    \item CMS-часть.
\end{enumerate}

Пользовательская часть приложения предназначена для использования потенциальными клиентами интернет магазина и должны предоставлять возможность:

\begin{itemize}
    \item Регистрации и авторизации на ресурсе;
    \item Просмотр информации об организации;
    \item Просмотр категорий товаров, товаров и их характеристик;
    \item Наполнение корзины;
    \item Оформление заказа;
    \item Контакт с менеджером;
    \item Просмотр и изменение личной информации в личном кабинете.
\end{itemize}

CMS-часть предназначена для использования менеджерами и администратором.
Контроль доступа к этой секции осуществляется сервером.
Обычный, анонимный пользователь или пользователь с недостаточным уровнем доступа, не может попасть на данную секцию приложения.
Основные возможности CMS-части:

\begin{itemize}
    \item Просмотр/добавление/изменение/удаление категорий товаров;
    \item Просмотр/добавление/изменение/удаление товаров, а также изменение списка изображение товара;
    \item Просмотр/добавление/изменение информации о зарегистрированных пользователях, а также изменение их скидок;
    \item Формирование и рассылка почтовых сообщений всем клиентам;
\end{itemize}

\subsubsection{Общий функционал Android-клиента}\indent

Поскольку разрабатываемое приложение должно использоваться менеджерами организации, приложение должно обеспечивать возможность авторизации пользователя с помощью установленных на удалённом сервере авторизационных данных. Неавторизованный пользователь не должен иметь возможности получить какие-либо данные из приложения, поскольку это может привести к раскрытию коммерческой тайны.

Авторизованные пользователи должны иметь возможность, в зависимости от уровня доступа:

\begin{itemize}
    \item Просмотр и изменение категорий товаров
    \item Просмотр и изменение полного списка товаров и их детальные страницы;
    \item Просмотр списка заказов, а также детальной информации по каждому из них;
    \item Просмотр, редактирование и добавление пользователей, редактирование их скидок;
    \item Изменение локальной конфигурации приложения;
    \item Формирование отчетов по определённым критериям
\end{itemize}
    \subsection{Проектирование общей архитектуры приложения}\label{subsec:2-design-architecure}\indent

Разрабатываемое приложение является приложением-сервером в клиент-серверной архитектуре.
Как следствие, для данной архитектуры необходимо использовать технологии, которые способствуют эффективной реализации всех поставленных задач.

\addimghere{app-architecture}{1}{Общая архитектура приложения}

Для реализации сервера будет использован \hyperlink{gloss:rest}{REST API} подход к реализации архитектуры.

Внутренняя архитектура \hyperlink{gloss:rest}{rest-сервера} и web-клиента будет следовать архитектурному паттерну MVC (Model – View – Controller).
В связи с особенностями платформы, Android-клиент будет реализован при помощи паттерна MVVM (Model – View – ViewModel).

\subsubsection{Архитектурный паттерн MVC}\indent

\addimghere{arch-mvc}{0.5}{Схема архитектурного паттерна MVC}

MVС – архитектурный паттерн проектирование позволяет разделить приложение на 3 связанные части.
Данный паттерн позволяет выделять из больших компонентов части, которые могут быть пере использованы и способствуют параллельной разработке.

Функциональные слои MVC:
\begin{enumerate}
    \item Model – представляет собой слой данных и реагирует на инструкции контроллера, изменяя своё состояние;
    \item View – отвечает за отображение данных модели пользователю, реагируя на изменение модели;
    \item Controller – принимает ввод и преобразует его в инструкции для модели или представления.
\end{enumerate}

\subsubsection{Архитектурный паттерн MVVM}\indent

\addimghere{arch-mvvm}{0.5}{Схема архитектурного паттерна MVVM}

MVVC – архитектурный паттерн проектирование позволяет разделить приложение на 3 отдельные части. Данный паттерн позволяет выделять из больших компонентов части, которые могут быть пере использованы и способствуют параллельной разработке.

Функциональные слои MVVM:
\begin{enumerate}
    \item Model – представляет собой слой данных.
    Обычно являются структурами или простыми Data классами;
    \item View – отвечает за отображение данных модели пользователю, реагируя на изменение модели.
    Является подписчиком на событие изменения значений свойств или команд, предоставляемых ViewModel.
    В случае, если в ViewModel изменилось свойство, она оповещает об этом своих подписчиков;
    \item ViewModel – содержит Model, преобразованную к View, а также команды, которыми может пользоваться View, чтоб влиять на модель.
\end{enumerate}
    \subsection{Проектирование диаграммы вариантов использования}\label{subsec:design-use-case}\indent

Диаграмма вариантов использования представлена на рисунке \ref{use-case}.

\addimghere{use-case}{0.7}{Диаграмма вариантов использования}

Согласно данной диаграмме, выделяется 3 роли:
\begin{enumerate}
    \item Клиент – обычный пользователь сайта. 
    Имеет возможность идентификации и аутентификации, просмотра каталога, наполнения корзины в допустимом количестве и оформление заказа.
    \item Менеджер – пользователь, занимающийся управлением контентом и актуализацией информации на сайте. 
    Имеет возможности по авторизации со своими пользовательскими данными, управления товарами (добавление/редактирование параметров), пользователями (регистрация новых/редактирование скидок).
    \item Администратор – пользователь с неограниченными возможностями по управлению содержимым базы данных.
\end{enumerate}
    \subsection{Проектирование БД}\label{subsec:design-db}\indent

Спроектированная диаграмма \hyperlink{gloss:db}{БД} представлена в Приложении \ref{addition:rest-database}.

На данной диаграмме, можно выделить несколько видов связей между таблицами:

\begin{enumerate}
    \item <<один-ко-многим>> – примером такой связи может выступать связь между сущностями "Категория" и "Товар", поскольку, одновременно товар может находиться только в одной категории.
    \item <<многие-ко-многим>> – пеализуется путём применения связей 1-к-многим через промежуточную таблицу. Так, например, сущность "Товар" связана связью многие-ко-многим с сущностью "Цвет",
    поскольку у каждого товара может быть множество цветов, так и каждый из цветов может быть связан со множеством товаров. 
    \item <<без связи>> – технические, узконаправленные таблицы. К ним относятся таблицы необходимые для хранения информации необходимой для идентификации и аутентификации пользователей в системе, 
    а также таблица содержащая в себе заранее определённые шаблоны почтовых сообщений с темами сообщений.
\end{enumerate}
    \subsection{Проектирование макетов основных экранов приложения}\label{subsec:2-design-mockups}\indent

\subsubsection{Мокапы Web-клиента}\indent

\pageblock{
    Шаблон главной страницы представлен на рисунке \ref{mockup-web-main}.
    \addimghere{mockup-web-main}{0.7}{Шаблон главной страницы приложения}
}

Тут можно выделить несколько элементов:
\begin{itemize}
    \item навигационное меню. Общее для всех экранов приложения.
    \item несколько блоков содержащих в себе краткую информацию об организации.
    \item блок с формой обратной связи и картой, по которой можно определить расположение офиса организации.
\end{itemize}

\pageblock{
    \bigskip
    Шаблон страницы каталога представлен на рисунке \ref{mockup-web-catalog}.
    \addimghere{mockup-web-catalog}{0.7}{Шаблон страницы каталога приложения}
}

Компонент товара в каталоге представлен блоком состоящим из:

\begin{itemize}
    \item галереи изображений товара.
    \item блока основной информации о товаре (артикул, заголовок, характеристики, цена).
    \item кнопка добавления в корзину.
\end{itemize}

\pageblock {
    \bigskip
    Шаблон страницы оформления заказа и корзины представлен на рисунке \ref{mockup-web-order}.
    \addimghere{mockup-web-order}{0.7}{Шаблон страницы оформления заказа и корзины}
}

Страница оформления заказа содержит:

\begin{itemize}
    \item блок просмотра и редактирования заказа:
    \begin{itemize}
        \item изображение товара;
        \item краткая информация о товаре, его цена, кол-во упаковок;
        \item переключатель количества упаковок в корзине;
        \item кнопка удаления позиции из корзины.
    \end{itemize}
    \item форму оформления заказа.
\end{itemize}

\subsubsection{Мокапы Android-клиента}\indent

\pageblock {
    Шаблон страницы авторизации в приложении представлен на рисунке \ref{mockup-android-auth}.
    \addimghere{mockup-android-auth}{0.25}{Шаблон страницы авторизации приложения}    
}

\pageblock{
    \bigskip
    Шаблон страницы каталога представлен на рисунке \ref{mockup-android-catalog}.
    \addimghere{mockup-android-catalog}{0.25}{Шаблон страницы каталога приложения}    
}

\pageblock{
    \bigskip
    Шаблон детальной страницы товара представлен на рисунке \ref{mockup-android-titlecard}.
    \addimghere{mockup-android-titlecard}{0.2}{Шаблон детальной страницы товара}
}
    \subsection{Описание технологий, используемых в разработке}\label{subsec:2-tech-review}\indent

% ## REST

\subsubsection{Технологии, использующиеся при разработке REST-сервера}\indent

В качестве языка программирования использован Kotlin.

Kotlin – статически типизированный язык програмиирования, работающий поверх \hyperlink{gloss:jvm}{JVM} и разрабатываемый компанией JetBrains. Имеет возможность компиляции в Javascript, а также ряд других платформ, через инфраструктуру LLVM. Авторы языка, ставили целью создание более лаконичного и типобезопасного, чем Java и более простого чем Scala языка.\cite{kotlin}

К достоинствам относят:
\begin{itemize}
    \item лаконичность языка;
    \item возможность создания расширений для типов, именнованые аргументы и ряд других фич, которые относят к разряду “cинтаксического сахара”;
    \item Kotlin официально поддерживается Google;
    \item полностью совместим с Java;
    \item при работающем проекте на Java, имеется возможность не переписывать всё на Kotlin, а лишь дописывать новый функционал, без нарушения работы в продукте.
\end{itemize}

К недостаткам можно отнести, достаточно малое сообщество разработчиков, однако оно постоянно расширяется.

Основным фрейворком является Spring Boot,который является упрощенной версией фрейворка Spring.

Spring – один из наиболее популярных фрейворков для разработки приложений для Java (на текущий момент заявлено, что Spring полностью совместим с Kotlin).
К основным особенностям фреймворка относят встроенная поддержка \hyperlink{gloss:di}{DI}, которая позволяет придерживаться принципа \hyperlink{gloss:ioc}{IoC}.
Spring помогает свободно разрабатывать полноценные приложения, которые достаточно просто покрываются юнит-тестами.

Spring boot – является упрощенной версией Spring фреймворка.
Spring boot позволяет взять на себя часть рутины связанной с конфигурацией проекта.

Spring security и Spring oauth2 – позволяют контролировать доступ к методам приложения, а также позволяет производить авторизацию и регистрацию пользователей.

Spring Data \hyperlink{gloss:jpa}{JPA} – реализует слой доступа к данным и призван значительно упростить реализацию слоя доступа к данным, сократив усилия на этом этапе и направив в области, которые действительно необходимы.
Достоинства:
\begin{itemize}
    \item поддержка репозиториев, основанных на Spring и \hyperlink{gloss:jpa}{JPA};
    \item поддержка типобезопасных \hyperlink{gloss:jpa}{JPA} запросов;
    \item прозрачный аудит для доменных классов;
    \item поддержка разбивки на страницы;
    \item возможность интеграции собственного кода для доступа к данным.
\end{itemize}

Для сборки проекта и управления зависимостями использован Gradle.

Gradle – открытая система для автоматизации сборки проектов.
Поддерживает инкрементальную сборку и может определять, какая часть древа была обновлена.
Одним из крупнейших преимуществ Gradle по сравнению с другими системами сборки(Maven, Ant и т.д.) является общая гибкость в настройках сборки и каталогов, 
без необходимости следовать ограничениям системы сборки.

Для написания Unit-тестов использована библиотека JUnit, которая является библиотекой для модульного тестирования \hypertarget{gloss:software}{ПО}.
Изначально данная библиотека была разработана для Java языка. Однако Kotlin полностью совместим с Java, поэтому JUnit может использоваться и для написания тестов для языка Kotlin.

Для общей гибкости при написании тестов использованы библиотеки Mockk и Assertj.

В качестве \hyperlink{gloss:db}{БД} использована реляционная \hyperlink{gloss:db}{БД} MySQL. В реляционной \hyperlink{gloss:db}{БД} данные хрянятся в таблицах.
Взаимосвязанные данные могут группироваться в таблицы, а также между таблицами могут быть установлены взаимоотношения.
К безусловным достоинствам данной \hyperlink{gloss:db}{БД} является контроль доступа, масштабируемость.

% ## WEB

\subsubsection{Технологии, использующиеся при разработке Web-клиента}\indent

Для вёрстки web-страниц использован язык разметки HTML. Для предания страницам дизайна, использован CSS.

HTML (от англ. HyperText Markup Language – «язык гипертекстовой разметки») – стандартизированный язык разметки документов во Всемирной паутине.
Большинство веб-страниц содержат описание разметки на языке HTML (или XHTML).
Язык HTML интерпретируется браузерами;
полученный в результате интерпретации форматированный текст отображается на экране монитора компьютера или мобильного устройства.
Текстовые документы, содержащие разметку на языке HTML (такие документы традиционно имеют расширение .html или .htm), обрабатываются веб-браузерам, которые отображают документ в его форматированном виде, предоставляя пользователю удобный интерфейс для запроса веб-страниц, их просмотра (и вывода на иные внешние устройства) и, при необходимости, отправки введённых пользователем данных на сервер.\cite{web-technologies}

CSS (от англ. Cascading Style Sheets – каскадные таблицы стилей) – формальный язык описания внешнего вида документа, написанного с использованием языка разметки.
Преимущественно используется как средство описания, оформления внешнего вида веб-страниц, написанных с помощью языков разметки HTML и XHTML, но может также применяться к любым XML-документам, например, к SVG или XUL. CSS используется создателями веб-страниц для задания цветов, шрифтов, расположения отдельных блоков и других аспектов представления внешнего вида этих веб-страниц. Основной целью разработки CSS являлось разделение описания логической структуры веб-страницы (которое производится с помощью HTML или других языков разметки) от описания внешнего вида этой веб-страницы (которое теперь производится с помощью формального языка CSS).\cite{web-technologies}

Фрейворк для web-приложения – Angular.

Angular – это открытая и свободная платформа для разработки веб-приложений, написанная на языке TypeScript, разрабатываемая командой из компании Google, 
а также сообществом разработчиков из различных компаний.
Предназначена для разработки одностраничных приложений.
Цель использования — расширение браузерных приложений на основе MVC-шаблона, а также упрощение тестирования и разработки.

Фреймворк работает с HTML, содержащим дополнительные пользовательские атрибуты, которые описываются директивами, и связывает ввод или вывод области страницы с моделью, представляющей собой обычные переменные JavaScript.
Значения этих переменных задаются вручную или извлекаются из статических или динамических JSON-данных.

Двустороннее связывание данных в Angular является наиболее примечательной особенностью, и уменьшает количество кода, освобождая сервер от работы с шаблонами.
Вместо этого, шаблоны отображаются как обычный HTML, наполненный данными, содержащимися в области видимости, определённой в модели.
Специальный сервис в Angular следит за изменениями в модели и изменяет раздел HTML-выражения в представлении через контроллер.
Кроме того, любые изменения в представлении отражаются в модели.
Это позволяет обойти необходимость манипулирования DOM и облегчает инициализацию и прототипирование веб-приложений.\cite{web-angular}

TypeScript — язык программирования, представленный Microsoft и позиционируемый как средство разработки веб-приложений, расширяющее возможности JavaScript.
TypeScript является обратно совместимым с JavaScript и компилируется в последний.
Фактически, после компиляции программу на TypeScript можно выполнять в любом современном браузере или использовать совместно с серверной платформой Node.js.
Код экспериментального компилятора, транслирующего TypeScript в JavaScript, распространяется под лицензией Apache.
Его разработка ведётся в публичном репозитории через сервис GitHub.
TypeScript отличается от JavaScript возможностью явного статического назначения типов, поддержкой использования полноценных классов (как в традиционных объектно-ориентированных языках), а также поддержкой подключения модулей, что призвано повысить скорость разработки, облегчить читаемость, рефакторинг и повторное использование кода, помочь осуществлять поиск ошибок на этапе разработки и компиляции, и, возможно, ускорить выполнение программ \cite{web-typescript}.

Angular Material состоит из набора предустановленных компонентов Angular.
В отличие от Bootstrap, предоставляющего компоненты, которые вы можете использовать любым способом, Anglate Material стремится обеспечить расширенный и последовательный пользовательский интерфейс.
В то же время он дает возможность контролировать, как ведут себя разные компоненты.

Material Design – это язык дизайна для веб и мобильных приложений, который был разработан Google.
Material Design упрощает разработчикам настройку \hyperlink{gloss:ui}{UI}, сохраняя при этом удобный интерфейс приложений.

% ## ANDROID

\subsubsection{Технологии, использующиеся при разработке Android-клиента}\indent

Для разработки нативного Android-приложения использован Android-фреймворк использован язык программирования Kotlin.

В разработке использованы элементы из Android Jetpack Architecture Components:
\begin{itemize}
    \item LiveData – хранилище данных, работающее по принципу паттерна Observer, которое умеет определять активность подписчика;
    \item LifeCycle – компонент для удобной работы с LifeCycle Activity;
    \item Android Ktx – функции расширения для стандартной библиотеки Android;
    \item Navigation – компонент облегчающий навигацию между фрагментами Android приложения;
    \item Room – ORM система для SQLLite;
    \item ViewModel – компонент позволяющий корректно обрабатывать состояние фрагмента или активити при изменении состояния (например, при повороте).
\end{itemize}

В качестве \hyperlink{gloss:di}{DI} фреймворка выступает Koin.
Koin – небольшая библиотека для внедрения зависимостей.
В отличии от большей части подобных библиотек, Koin не использует кодогенерацию, проксировани или итроинспецкию.
Из дополнительных плюсов, Koin использует \hyperlink{gloss:dsl}{DSL} и функционал языка Kotlin.
Подразумевается использования с Kotlin, однако, Java тоже может работать вместе с Koin.

OkHttp использован для реализации возможности выполнения сетевых запросов и сетевого взаимодействия.
Эта библиотека обладает полным функционалом для работы с любым \hyperlink{gloss:rest}{REST API}, легко тестируется и настаивается.

    \subsection{Описание основных компонентов приложений}\label{subsec:2-common-components}\indent

% ## WEB

\subsubsection{Основные компоненты веб-клиента}

К основным страницам разрабатываемого приложения относятся:

\begin{itemize}
    \item главная страница;
    \item каталог;
    \item корзина/Форма оформления заказа;
    \item страница “Товары” \hyperlink{gloss:cms}{CMS}-части;
    \item страница “Пользователи” \hyperlink{gloss:cms}{CMS}-части.
\end{itemize}

Разберём основной функционал, который должны предоставлять данные страницы.

Главная страница – является отправной точкой для пользователя и содержит основную информацию об организации, владеющей интернет магазином.
По сути является Landing-page.

Каталог – содержит список товаров, по категориям, которые отсортированы по наличию и цене.
Каждый товар обладает своим рядом характеристик, а также изображениями.
Некоторые товары могут обладать цветами, в таком случае, в корзину складывается не просто товар, а еще и его цвет.
Изображения каждого из товаров переключается с заданным интервалом.
В случае, отсутствия изображений, блок с изображениями заменяется на fallback-image.

Корзина/Форма оформления заказа – предоставляет возможность изменить кол-во товара и их список, которые будут использованы при оформлении заказа.
Каждый товар обладает рядом осноных характеристик и одним изображением.
Цена за позицию должна изменяться на лету, в зависимости от кол-ва товаров в корзине.
Форма оформления заказа должна поддерживать валидацию введённых данных, перед отправкой запроса на сервер.
В случае успешного оформления заказа, происходит переадресация на главную страницу приложения и очистки локальной корзины.

Страница “Товары” \hyperlink{gloss:cms}{CMS}-части – страница администраторской части приложения, которая доступна только пользователями с уровнем доступа Manager и выше.
Контроль доступа осуществляется сервером.
На данной странице есть возможность просмотра/добавления/редактирования и изменения категорий товаров и товаров.
Товары сгруппированы по категориям.
Имеется возможность быстрой установки информации о том, что товар отсутствует на складе или удалён.
Кроме того, присутствует возможность изменения информации о товаре и его изображения на специальной форме.
Для добавления товара используется отдельная форма.

% ## ANDROID

\subsubsection{Основные компоненты Android-клиента}

К основным страницам разрабатываемого приложения относятся:

\begin{itemize}
    \item cтраница авторизации;
    \item cписок категорий и их продуктов;
    \item cтраница детальной информации о товаре;
    \item cписок заказов и информация о них;
    \item cписок зарегистрированных пользователей, с возможностью детального просмотра информации, а также удалении/добавления.
\end{itemize}

Разберём основной функционал, который должны предоставлять данные страницы.

Контроль доступа осуществляется со стороны сервера.

Страница авторизации – является отправной точкой для пользователя и содержит небольшое приветственное сообщение и поля для ввода авторотационных данных.

Список категорий и их продуктов содержит список товаров, по категориям.
Каждый товар обладает своим рядом характеристик, а также сопровождается изображениями.
Некоторые товары могут обладать цветами.
Изображения каждого из товаров переключются с заданным интервалом.
В случае, отсутствия изображений, блок с изображениями заменяется на fallback-image.
На данной странице есть возможность просмотра/добавления/редактирования и изменения категорий товаров и товаров.
Товары сгруппированы по категориям.
Имеется возможность быстрой установки информации о том, что товар отсутствует на складе или удалён.
Кроме того, присутствует возможность изменения информации о товаре и его изображения на специальной форме.
Для добавления товара используется отдельная форма.

    \subsection{Диаграмма деятельности некоторых функций приложения}\label{subsec:2-activity-diagram}\indent

% ## WEB

\subsubsection{Диаграмма деятельности Web-клиента}\indent

Диаграмма деятельности Web-клиента представлена на рисунке \ref{diagram-activity-web_2}
\clearpage
\begin{sidewaysfigure}
    \addimghere{diagram-activity-web_2}{0.87}{Диаграмма деятельности Web-клиента}

\end{sidewaysfigure}
\clearpage
% % \addimghere{diagram-activity-web_2}{1}{Диаграмма деятельности Web-клиента}

Согласно представленной диаграмме, после авторизации пользователя в приложении, в зависимости от роли, которая закреплена за данным пользователем, изменяется список функционала, 
который доступен пользователю.
Так, например, анонимный пользователь, имеет возможность зарегистрироваться/авторизоваться в приложении, в последствии получив одну из ролей, которая закреплена за данным пользователем.
Помимо авторизации и регистрации, пользователь имеет возможность просматривать каталог товаров, наполнять корзину и делать заказ.

В случае, если пользователь обладает уровенем доступа выше уровня "Менеджер", включительно, для него открывается доступ в \hyperlink{gloss:cms}{CMS} часть сайта, 
в которой есть возможность редактировать список товаров, категорий, пользователей и т.д.

% ## ANDROID

\subsubsection{Диаграмма деятельности Android-клиента}\indent

Диаграмма деятельности Android-клиента представлена на рисунке \ref{diagram-activity-android_2}

\clearpage
\begin{sidewaysfigure}
    \addimghere{diagram-activity-android_2}{0.87}{Диаграмма деятельности Android-клиента}

\end{sidewaysfigure}
\clearpage

Диаграмма деятельности Android-клиента имеет схожую структуру и суть с диаграммой Web-клиента, за исключением невозможности работы неавторизованным пользователем 
и невозможностью пользоваться функционалом предназначенным для простого пользователя приложения (например, складывать товары в корзину или оформлять заказ). 
Однако, пользователю Android-приложения доступен весь функционал, доступный менеджерам и администраторам CMS-части Web-клиента. 


    \subsection{Вывод по главе 2}\label{subsec:2-conclusion}\indent

    Во второй главе описаны основные этапы разработки приложения, функционал, 
    а также перечислены основные технологии и фреймворки, которые будут использованы в процессе разработки. Представлена диаграмма вариантов использования, 
    схема \hyperlink{gloss:db}{БД}, диаграммы деятельности, макеты основных экранов и сформированы требования для каждого из экранов разрабатываемых клиентских приложений.

    % #### CHARPTER 3 - Implementation ####
    \section[Глава 3 Реализация системы приложений интернет-магазина]{Глава 3\break Реализация системы приложений интернет-магазина}
    \label{sec:charpter-3-inplementation}

    \subsection{Реализация приложения REST-сервера}\label{subsec:3-impl-server}\indent

\subsubsection{Описание архитектурных решений}\indent

При реализации использовался архитектурный паттерн MVC. Данный паттерн позволяет разделить данные, представление и бизнес-логику.

Все классы были разделены на 3 слоя:
\begin{enumerate}
    \item Данные – в этот слой входят все классы из пакетов model и repository
    \item Сервис – в этот слоя входят классы из пакета service.
    На этом уровне выполняется основная бизнес-логика приложения.
    Сервис существует для каждой значимой сущности и абстрагирован от других сущностей.
    \item Контроллер – в этот слой входят классы из пакета controller.
    На данном слое происходит обработка ошибок и формирования ответов клиенту.
    На данном слое происходит json-сериализация с помощью встроенного в Spring-framework сериализатора Jackson.
    Для получения/отправки данных используется шаблон проектирования DTO (Data Transfer Object).
\end{enumerate}
Отдельными модулями приложения являются пакеты:

\begin{enumerate}
    \item Utils – пакет в котором хранятся общие утилиты, необходимые приложению, а также поддерживаемые Kotlin-ом расширяющие функции
    \item Config – пакет в котором производится конфигурация Spring фреймворка.
    А также создаваемые для Spring DI – компоненты. Примеры конфигурации бинов будут приведены в приложении
\end{enumerate}

\subsubsection{Описание основных аннотаций}\indent

Для маппинга Kotlin data классов используемых в БД используются JPA Persistence API аннотации.

@Entity – используется для сообщения Spring о том, что класс является сущностью используемой в БД.

Т.к. для каждого класса, используемого в БД необходим, конструктор по умолчанию (без параметров), а kotlin-data классы не поддерживают его, в приложении используется kotlin-noarg gradle плагин, который генерирует конструктор по умолчанию для всех data классов.

@Table – используется для указания имени таблицы, которое будет использовано при обращении.

@get: - является решением, для того чтоб размещать аннотации над соответствующим геттером(т.к. в Kotlin геттеры не пишутся и генерируются автоматически).

@Column – используется для указания имени, а также некоторого списка характеристик колонки таблицы (например nullable) определённого поля класса.

Если в классе есть поля объектного типа, то должна быть указана связь между таблицами (@OneToMany, @ManyToOne, @ManyToMany), а так же аннотация для указания, по какому полю производить связь (@JoinColumn).

@Controller, @RestController, @Service, @Repository – используются для указания Spring о том, к какому типу компонента относится тот или иной класс.

@Bean – сообщает Spring о том, что объект является Spring-Bean.

@Autowired – сообщает Spring о том, что реализацию данного поля нужно найти среди Spring-Beanов.

@Configuration – указывает на то, что класс является конфигурационным.

Связка @JsonManagedReferense/@JsonBackReference является аннотациями Jackson-сериализатора и служат для того, чтоб избежать рекурсивной десериализации объектов, которые ссылаются друг на друга.

@JsonIgnore – служит для того, чтоб указать Jackson-сериализатору то, что данное поле следует игнорировать пре сериализации.

@GetMapping, @PostMapping, @PutMapping, @DeleteMapping, @RequestMapping – аннотации сообщающие диспатчеру о том, по какому url адресу, данный метод должен обрабатывать запросы.

\subsubsection{Описание процесса авторизации}\indent

Для контроля доступа и возможности назначения персональных скидок, в системе реализована возможность авторизации. Авторизация происходит по протоколуOAuth2

OAuth – открытый протокол авторизации, который позволяет предоставить третьей стороне ограниченный доступ к защищенным ресурсам пользователя без необходимости передавать ей логин и пароль.

Преимущества OAuth2:
\begin{itemize}
    \item Клиент может быть уверен в том, что несанкционированный доступ к его личным данным невозможен.
    Не владея логином и паролем пользователя, приложение может выполнять только ограниченный ряд действий
    \item Не нужно заботиться об обеспечении конфиденциальности логина и пароля.
    Т.к. логин и пароль не передаются приложению и как следствие, не могут попасть в руки злоумышленников
\end{itemize}

Результатом авторизации является получение access token – некий ключ (хешированая строка) предъявление которого является доступом к защищенным ресурсам.
Самым простым способом передачи является его указание в заголовках вместе с запросом.

\subsubsection{Описание работы почтового клиента с генерацией шаблонов}\indent

Для отправки e-mail сообщений используется Java-класс JavaMailSender, который сконфигурирован как Spring Bean и доступен для Dependecy Injection.
Для конфигурации данного класса ему передаются список параметов (как например логий и пароль от SMTP сервера, который будет рассылать сообщения)

На текущий момент сообщения отправляются при оформлении заказа.
Т.е. когда приходит запрос на регистрацию заказа, заказ сначала добавляется в БД, в случае успешного добавления, из базы данных получается необходимый шаблон почтового сообщения, в который вставляются данные заказа.
Для вставки корректных данных, в шаблоне предусмотрены специальные метки заданные регулярным выражением: "<\textbackslash\textbackslash[([\textasciicircum\textbackslash\textbackslash]\%]*)\textbackslash\textbackslash]>"

\subsubsection{Unit-тестирование}\indent

Unit-тестирование – процесс позволяющий проверить на корректность отдельные модули исходного кода программы, с соответсвующиеми управляющими данными, процедурами использования и обработки. Идея заключается в том, чтобы писать тесты для каждой нетривиальной функции или метода. Это позволяет достаточно быстро проверить работоспособность кода и не привело ли изменение к регрессии.

Для упрощения написания тестов, в приложении используются библиотеки Mockk и AssertJ.

Mockk – это простая Kotlin библиотека для создания объектов заглушек, которые не несут в себе никакой логики, однако используются для симуляции поведения объектов с определёнными условиями. Т.к. эта библиотека работает по приципу наследования от мокируемого объекта, а в Kotlin все классы являются по умолчанию ненаследуемыми, необходимо использовать allopen graddle плагин, который делает все классы открытыми по умолчанию.

AssertJ – библиотека, которая предоставляет удобный интерфейс для написания тестовых сравнений и главной целью ставит улучшение читаемости тестового кода, а также повышения удобности отладки тестов.

В текущей системе, тестами покрыты все нетривиальные методы Сервис-слоя, а также сериализаторы и парсер почтовых сообщений.


\subsubsection{Сборка проекта}\indent

Для сборки проекта используется система автоматической сборки Gradlle.

Данный сборщик поддерживает написание build-скриптов на языке Kotlin, Groovy. Имеется возможность тонкой настройки скриптов сборки, а также дополнительных задач, которые будут выполнены перед сборкой.

Кроме того, есть возможность скачивания зависимостей из maven-repository сервисов и поддержка плагинов (в данном приложении, например были использованы плагины allopen и noargconstr, для обеспечения совместимости языка Kotlin с некоторыми библиотеками или фреймворками. Также, поддержка инкрементальной сборки и отслеживание изменённого кода, может значительно сократить время сборки (особенно это видно на больших проектах).

    \subsection{Реализация приложения Web-клиента}\label{subsec:3-impl-web}\indent

\subsubsection{Описание архитектурных решений}\indent

Для реализации выбран архитектурный паттерн MVC. Данный паттерн позволяет разделить данные, представление и бизнес-логику.

Все файлы приложения были разделены на 3 слоя:
\begin{enumerate}
    \item Данные – в этот слой входят все файлы, которые отвечающие за представление структур данных, которые используются в приложении.
    Все данные, которые приложение получает с сервера или с локального хранилища представлены в виде интерфейсов данного слоя.
    \item Сервис – на этом уровне выполняется основная бизнес-логика приложения и запросы к серверу.
    Сервис существует для каждого значимого функционала значимой сущности и соответствует \hyperlink{gloss:spr}{SPR (Single Responsibility Principle)}.
    \item Контроллер – на данном слое происходит обработка ошибок и биндинг данных в view, а также происходит управление состоянием view в зависимости от существующих данных, либо статуса загрузки данных с сервера.
\end{enumerate}

Отдельными модулями приложения являются:
\begin{itemize}
    \item Pipes – содержит классы, которые занимаются форматирование данных при отображении.
    Например, CountPipe, PricePipe
    \item Utils – содержит классы, утилиты и икапсулированные обёртки вокруг библиотек, настроенные для использования в реализуемом приложении.
    \item Routing – содержит всю логику и весь маппинг возможных переходов по приложению.
\end{itemize}

\subsubsection{Описание основных сервисов}\indent

В приложении используется аутентификация на сервере по технологии OAuth2, поэтому должна быть реализована логика, 
которая может быть легко встраиваемой в любой компонент при помощи \hyperlink{gloss:di}{DI}. 
Кроме того, должны быть механизмы перезапроса access\_token’а, при наличии refresh\_token’а, в случае истечения срока его действия.
Для этого был реализован AuthorizationService, листинг которого можно увидеть в Приложении \ref{addition:web-authservice}.
Данный сервис инкапсулирует в себе логику для контроля авторизационных процессов.
Поскольку данный класс реализует в себе HttpInterceptor интерфейс, он может быть добавлен как перехватчик к любому исходящему запросу и выполнять необходимую логику перезапроса токена, при наличии refresh\_token’а и получении 403 ошибки при выполнении запроса и добавления токенов в заголовки запроса, при их наличии.
Все токены хранятся в local storage браузера.

CartService занимается контролем за состоянием корзины, а также её управлением.

На каждый из контроллеров сервера, реализованы свои сервисы.

Основная их задача заключается в выполнении запроса к серверу и возвращение подписки на \hyperlink{gloss:ui}{UI}.
При помощи данного callback’а имеется возможность выполнения запросов в сеть без блокирования UI потока.

Пример реализации MgrProductService можно найти в Приложении \ref{addition:web-mgr-product-service}.

Пример реализации подписки на получаемый результат от сервиса представлен на листинге \ref{lis:web-subscription}.

\begin{lstlisting}[language=TypeScript, captionpos=b,
label={lis:web-subscription},
caption={Пример реализации подписки на ожидаемый результат от сервера}
]
private loadProducts(id: number) {
    this.productService.getByCategoryId(id).subscribe(
    products => {
        this.tableConfig.source = new MatTableDataSource<Product>(products);

        setTimeout(() => {
            ProductComponent.scrollToView(this.productActionsSubSection);
            this.tableConfig.source.sort = this.sort;
            this.tableConfig.source.paginator = this.paginator;
        });
    }, error => {
        this.processError(error);
    }, () => this.setLoading(false)
    )
}
\end{lstlisting}

\subsubsection{Описание принципов построения пользовательского интерфейса}\indent

В Angular пользовательский интерфейс состоит из легко встраиваемых компонентов.
Каждый компонент создаётся разработчиком и может управлеяет отображением представления на экране.
Для создания компонента необходимо импортировать функцию декоратора @Component из библиотеки @angular/core.
Данный декоратор позволяет идентифицировать класс как компонент.

Декоратор в качестве параметра принимает объект с конфигурацией, которая указывает фреймворку, как работать с компонентом и его представлением.
С помощью свойства template, шаблон представляет часть HTML разметки с вставкой кода Angular.
Фактически, шаблон и является представлением, которым пользователь управляет при работе с приложением.
Каждый компонент должен обладать одним шаблоном.
Свойство selector определяет селектор CSS. В элемент с этим селектором Angular будет добавлять представление компонента.

Некоторые элементы форм клиентской части и вся \hyperlink{gloss:cms}{CMS}-часть приложения используют Angular Material Components.

Для обеспечения адаптивности приложения используется CSS-Grid Layout.
Данный подход позволяет менять расположение grid элементов не меняя сам HTML. К основным понятиям CSS Grid относят:
\begin{itemize}
    \item Grid container – набор пересекающихся горизонтальных и вертикальных grid линий, которые делят пространство контейнера на области, в которые могут быть помещены grid элементы.
    \item Grid lines – это горизонтальные и вертикальные разделители grid контейнера.
    Эти линии находятся по обе стороны от столбца или строки.
    Разработчик может задать для данного элемента имя или числовой индекс, которые может использовать дальше в стилях.
    Нумерация начинается с единицы.
    Важный нюанс, данный элемент восприимчив к режиму написания, который используется на вашем ресурсе.
    Например, вы используете Арабский язык или любой другой язык у которого режим написания справа налево, то нумерация линий будет начинаться с правой стороны.
    \item Grid track – это пространство между двумя смежными grid линиями, вертикальными или горизонтальными.
    \item Grid cell – это наименьшая неделимая единица grid контейнера на которую можно ссылаться при позиционировании grid элементов.
    Образуется на пересечении grid строки и grid колонки.
    \item Grid area – это пространство внутри grid контейнера, в которое может быть помещен один или больше grid элементов.
    Этот элемент может состоять из одной или более grid ячеек.
\end{itemize}

Каждый элемент тесно связан друг с другом и отвечает за определенную часть grid контейнера.

Пример HTML для ProductAdd компонента можно найти в Приложении \ref{addition:web-product-add}.
CSS для этого компонента находится в Приложении \ref{addition:web-product-add-css}.

\subsubsection{Описание основных сторонних библиотек}\indent

Основные сторонние библиотеки, используемые в приложении:
\begin{itemize}
    \item Ngx-gallery – библиотека предоставляющая компонент для простой реализации автоматической галлереи изображений, обладающая рядом дополнительных функций.
    Используется на странице каталога товаров.
    \item Ngx-infinite-scroll – библиотека предоставляющая возможность порционной загрузки данных по мере приблежения к концу страницы.
    Используется на странице каталога товаров.
    \item RxJs – библиотека для обеспечения возможности реактивного программирования.
    \item Angular4-carousel – библиотека предоставляющая слайдер компонент используемый на главной странице приложения.
    \item Angular-notifier – библиотека предоставляющая настраиваемые всплывающие уведомления, которые используются в ответ на действия пользователя, по всей клиентской части приложения.
    \item Angular-2-local-storage – библиотека предоставляющая абстрактную обёртку вокруг local-storage, которая инкапсулирует всю логику работы с ним и предоставляет удобный интерфейс разработчику.
\end{itemize}

\subsubsection{Сборка и структура проекта}\indent

Для разработки приложения использовался Angular CLI – интерфейс командной строки, который позволяет быстро создавать проекты, добавлять файлы и выполнять множество определённых задач, таких как тестирование, сборка и развёртывание.
Для корректной работы Angular CLI, необходимо чтоб были установлены Node.js и npm.

Для запуска веб-сервера, используемого для разработки приложения необходимо выполнить комманду ng serve –open в дирректории Angular приложения.
Команда ng serve запускает веб-сервер, а также прослушивает каталог c исходниками приложения и при изменениях в этих исходных файлах пересобирает проект «на лету».
Стоит отметить, что в таком режиме проект не сохраняется на диске, он записывается непосредственно в оперативную память.
Использование ключа --open (или просто -o) означает, что после сборки проекта, автоматически откроется браузер (по умолчанию выбранный в операционной системе).

Пример структуры angular приложения представлен на листинге \ref{lis:web-project-structure}.

\begin{lstlisting}[language=TypeScript, captionpos=b,
label={lis:web-project-structure},
caption={Пример структуры angular приложения}]
.
|-- app
|   |-- app.component.css
|   |-- app.component.html
|   |-- app.component.spec.ts
|   |-- app.component.ts
|   `-- app.module.ts
|-- assets
|-- environments
|   |-- environment.prod.ts
|   `-- environment.ts
|-- favicon.ico
|-- index.html
|-- main.ts
|-- polyfills.ts
|-- styles.css
|-- test.ts
|-- tsconfig.app.json
|-- tsconfig.spec.json
`-- typings.d.ts
\end{lstlisting}

Исходники приложения, располагаются в директории src.

\textit{app/app.component.{ts,html,css,spec.ts}} – корневой компонент приложения в который внедряются все остальные компоненты приложения. В нём указан корневой компонент иерархии представления. 
Сопровождается html-шаблоном, css-стилями и юнит-тестами.

\textit{app/app.module.ts} – определяет корневой модуль AppModule, в котором определено как собирается приложение.

\textit{assets/*} – директория, в которой размещаются файлы ресурсов использующиеся в приложении. После сборки приложения, ресурсы копируются без изменений.

\textit{index.html} – главная HTML-страница, которая загружается при посещении пользователем. AngularCLI автоматически добавлет весь JavaScript код и CSS файлы при сборке.

\textit{main.ts} – главная точка входа Angular приложения. По умолчанию приложение компилируется JIT(Just In Time) компилятором и запускает его в браузере.

Пример структуры корневой директории проекта представлен на Листинге \ref{lis:web-root-structure}.

\begin{lstlisting}[language=TypeScript, captionpos=b,
label={lis:web-root-structure},
caption={Пример структуры корневой директории проекта}]
.
|-- README.md
|-- e2e
|-- karma.conf.js
|-- node_modules
|-- package-lock.json
|-- package.json
|-- protractor.conf.js
|-- src
|-- tsconfig.json
`-- tslint.json

\end{lstlisting}

\textit{node\_modules/} – окружение Node.js создает данную директорию, в которой хранятся все сторонние модули, подключаемые из-вне путём перечисления в package.json.

\textit{.angular-cli.json} – конфигурационный файл AngularCLI. Посредством этой конфигурации можно установить некоторые из значений сборки по умолчанию, 
а также конфигурировать список файлов которые будут использованы при сборке проекта.

\textit{.editorconfig} – конфигурационный файл редактора кода. Специфицирует конфигурацию форматирования текста кода, 
большинство современных редакторов кода поддерживает конфигурацию полученную из данного файла.

\textit{.gitignore} – файл контроля версий, содержит список файлов которые надо игнорировать при загрузке не в систему контроля версий Git-репозиторий.
К разряду не нужных файлов относятся файлы сгенерированные редактором кода библиотеками кодогенерации.

\textit{package.json} – конфигурационный файл npm, в нем перечисляются сторонние модули (пакеты) разработчиков, которые использует ваш проект.

\textit{tsconfig.json} – конфигурация компилятора TypeScript для редактора кода.

\textit{tslint.json} – конфигурация для статического анализатора TSLint, используется при запуске ng lint.

    \subsection{Реализация приложения Android-клиента}\label{subsec:3-impl-android}\indent

\subsubsection{Описание архитектурных решений}\indent

При реализации использовался архитектурный паттерн MVVM. Данный паттерн позволяет разделить данные, представление и бизнес-логику.

Все файлы приложения были разделены по пакетам-фичам (англ. Feature).
Каждая из таких фич имеет в себе строгую иерархию классов, которая помогает следовать Single Responsibility принципу:
\begin{enumerate}
    \item Fragment – слой View.
    Является отображением модели
    \item ViewModel – слой ViewModel.
    Хранит в себе объект LiveData и изменяет его в зависимости от каких-либо сценариев.
    Кроме того, содержит в себе один или несколько UseCase.
    В LiveData хранится объект состояния Фрагмента
    \item State – инкапсулирует в себе данные и состояние View (Например, Loading, DataReady и т.д.)
    \item UseCase – класс отвечающий за одно определённое действие.
    Например, получение списка всех товаров.
    В UseCase может быть внедрён один или несколько репозиториев.
    На данном уровне выполняется запуск и контроль корутин
    \item Repository – слой отвечающий за получение данных из каких-либо источников (сеть или локальная база данных в зависимости от ситуации).
    Обычно в себе содержит несколько api-классов, которые инкапсулируют в себе запросы на удалённый сервер и парсинг полученной модели и несколько dao-классов, которые инкапсулируют в себе получение данных из локальной базы данных
\end{enumerate}

Отдельными пакетами приложения являются:
\begin{itemize}
    \item Networking – содержит классы и api-интерфейсы, которые инкапсулируют в себе логику сетевых запросов
    \item Database - содержит классы, api-интерфейсы и модели данных которые инкапсулируют в себе логику запросов в базу данных
    \item CommonUtils – Небольшие утилитарные классы и функции-расширения
\end{itemize}

Поскольку в приложении используется аутентификация на сервере по технологии OAuth2, должна быть реализована логика, которая может быть легко встраиваема в любой компонент при помощи Dependency Injection, кроме того, должны быть механизмы перезапроса access\_token’а, при наличии refresh\_token’а, в случае истечения срока его действия.
Для был реализован AuthTokenInterceptor, листинг которого можно найти в Приложении \ref{addition:android-auth-interceptor}.
Данный сервис инкапсулирует в себе логику для контроля авторизационных процессов.
Поскольку данный класс реализует в себе Interceptor интерфейс, он может быть добавлен как перехватчик к любому исходящему запросу и выполнять необходимую логику перезапроса токена, при наличии refresh\_token’а и получении 403 ошибки при выполнении запроса и добавления токенов в заголовки запроса, при их наличии.
Все токены хранятся в sharedPreferences.

На каждый из контроллеров сервера, реализованы свои api-классы.
Основная их задача заключается в выполнении запроса к серверу и возвращение данных на уровень Репозитория для дальнейшей обработки или кеширования.

Для того, чтоб не блокировать UI поток во время выполнения сетевых запросов, используются Kotlin-корутины.
Контроль за созданием и переключением контекстов корутин находится на уровнях UseCase-Repository.

Пример реализации можно найти в Приложении \ref{addition:android-request-coroutine}

На уровне Fragment происходит подписка на изменение состояния ViewModel.
Пример реализации подписки на получаемый результат от ViewModel представлен на листинге \ref{lis:android-subscription}.

\begin{lstlisting}[language=Kotlin, captionpos=b,
label={lis:android-subscription},
caption={Пример реализации подписки на ViewModel}
]
productsListViewModel.model.observe(viewLifecycleOwner) {
    when (it) {
        is Loading -> onLoading()
        is NoData -> {
            onLoadingStopped()
            list_products.visibility = GONE
            text_no_data.visibility = VISIBLE
        }
        is DataReady -> {
            adapter.cleanAddAll(it.products)
            onLoadingStopped()
            text_no_data.visibility = GONE
            list_products.visibility = VISIBLE
        }
    }
}
\end{lstlisting}

\subsubsection{Описание процесса кэширования данных}\indent

Поскольку одним из требований разрабатываемого приложения является возможность работы без подключения к интернету, отображая последнюю полученную с сервера информацию, необходимо разработать механизм кэширования данных полученных вследствии Http запросов к серверу.

Кэширование - один из способов оптимизации приложений.
Его суть заключается в сохранении данных, которые были получена тяжеловесной операцией из какого-либо источника, в пямяти приложения/базе данных и т.д. для дальнейшего более быстрого его получения и обработки.
Кроме того, кеширование позволяет разгрузить нагрузку с сервера, поскольку к нему не будут выполняться запросы каждый раз, а только при истечении актуальности локального кэша, либо при принудительном запросе обновления данных.
В контексте мобильного приложения, кэширование позволяет создать offline режим работы приложения.
В таком случае, приложение будет иметь возможность отображать последние закэшированные данные даже без подключения к интернету.

Политики кеширования задаются полем Cache-Control общего заголовка HTTP. Http клиент OkHttp предоставляет, позволяющую автоматически котролирвать кэширование запросов.
Однако, в рамках дипломной работы, данная функция не будет использована и вся логика кеширования HTTP запросов будет выполнена самостоятельно.

Рассмотрим логику кеширования данных и проверки их актуальности.
\TODO{Интерцепторы, если успею}
Для этого, необходимо создать одно место, через которое будет проходить все попытки клиента загрузить данные и в случае их устаревания, разрешать выполнение запроса к серверу данных.
Таковым местом в приложении будет являеться объект HttpRequestManager.
Он будет зарегистрован в DI Koin, как single, что гарантирует то, что данный объект будет являться singleton'ом.
На вход, метод HttpRequestManager\#request
\begin{itemize}
    \item path - относительный путь, по которому надо выполнить запрос
    \item method - HTTP метод с которым будет выполнен запрос
    \item cacheControl - объект содежащий информацию о том, когда ранее загруженные данные перестанут быть актуальными.
    В случае, если этот параметр null, кеширование не будет произведено
    \item queryParams - параметры с которыми надо выполнять запрос
\end{itemize}

Для сохранения данных о том, когда в последний раз был выполнен какой-либо запрос, в локальной базе данных создана сущность RequestCache.
Которая содежит в себе информацию о запросе, параметрах, методе, пути и времени, когда были выполнены все запросы, которые необходимо кешировать.
Перед любым запросом к серверу и базы данных удаляются все неактуальные записи.
В случае, если при попытки выполнения запроса к серверу, в таблице RequestCache уже будет содержаться запись об актуальности запроса, то запрос не будет выполнен и данные будут прочитаны из БД.
Идентификация запросов выполняется по хэш-коду модели запроса.

Полный листинг HttpRequestManager класса с обработкой кэша расположен в Приложении \ref{addition:android-http-request-manager}

\subsubsection{Описание принципов построения пользовательского интерфейса}\indent

В Android пользовательский интерфейс состоит из легко встраиваемых компонентов.
Каждый компонент может быть создан разработчиком и может управлять отображением представления на экране.
Компоненты должны быть описаны в layout.xml файле.
Вся вёрстка происходит в xml.
Программист может встраивать в layout файл как заранее определённые компоненты, так и написанные самостоятельно.
В рамках курсовой работы, полностью новые View классы не были написаны.

View – базовый компонент для всех Android компонентов.
Кроме того, есть еще ViewGroup, который является базовым для всех компонентов, обладающих возможностью хранить в себе другие компоненты.

В рамках курсовой работы, для большей части компонентов, базовым использовался ConstraintLayout.
ConstraintLayout – достаточно новый вид layout, который создан для уменьшения кол-ва иерархий layout, что влияет на производительность.
ConstraintLayout позволяет располагать View друг относительно друга с помощью Constrains правил.

Для формирования списков в Android используются такие компоненты как ListView и RecyclerView.
Их различие заключается в том, что RecyclerView переиспользует View, которые вышли за границы экрана и не видимы пользователю, таким образом, экономя память и производительность устройства, поскольку даже для бесконечного списка, системой будет создано только то кол-во View, которое помещается на экран.
ListView подходит для формирования небольших списков.

Для конфигурации RecyclerView, ему необходимо передать layoutManager и adapter.

LayoutManager – класс ответственный за отображение элементов RecyclerView, за их пролистывание и размещение на экране

Adapter – класс –реализация паттерна проектирования Адаптер, является конвертером между данными и View.
К основным функциям Adapter’а относится onCreateViewHolder – создающая View для RecyclerView, onBindViewHolder – производит установку значений в созданную View.
Для создания новых View их необходимо создать из xml разметки используя LayoutInflater\#inflate.

Пример конфигурации Adapter для RecyclerView компонента можно найти в Приложении \ref{addition:android-recyclerview-adapter}.

\subsubsection{Описание использованных сторонних библиотек}\indent
Все зависимости используемые в проекте указаны в build.gradle файле приложения, который представлен в приложении.

Основные сторонние библиотеки, используемые в приложении:
\begin{itemize}
    \item Kotlin – стандартная библиотека Kotlin-функций
    \item Kotlin-coroutines – поддержка Kotlin-корутин
    \item Android Material – библиотека компонентов в MaterialDesign стиле
    \item Navigation – для облегчения навигации между фрагментами приложения. Помогает организовать удобную навигацию в Single Activity приложении.
    \item Koin – DI фремворк с поддержкой viewModel, scope и KotlinDSL
    \item Retrofit2 – библиотека инкапсулирующая логику сетевых запросов
    \item Room – ORM для SqlLite
\end{itemize}

\subsubsection{Сборка и структура проекта}\indent
Для разработки приложения использовался gradle – система автоматической сборки, построенная на принципах ApacheAnt и ApacheMaven, но предоставляющая DSL на языках Groovy и Kotlin.
Был разработан для расширяемых многопроектных сборок и поддерживает инкрементальные сборки, определяя какие компоненты дерева сборки не изменились и какие задачи, зависимые от этих частей, не требуют перезапуска.

Пример структуры android-приложения представлен на листинге \ref{lis:android-structure}

\begin{lstlisting}[language=TypeScript, captionpos=b,
label={lis:android-structure},
caption={Пример структуры Android приложения}
]
.
|-- app
|   |-- src
|   |   |-- androidTest
|   |   |-- main
|   |   |   |-- java
|   |   |   |-- res
|   |   |   `-- AndroidManifest.xml
|   |   `-- test
|   |-- build.gradle
|   `-- proguard-rules.pro
|-- build.gradle
|-- settings.gradle
|-- gradle.properties
|-- gradlew
`-- gradlew.bat
\end{lstlisting}

Исходники приложения, как правило, располагаются в директории src. В папке res располагаются все ресурсы проекта (строки, переводы, layoust, anim, drawable, navigation).

Файл proguard-rules.pro содержит конфигурацию обфускации кода.


    \subsection{Вывод по главе 3}\label{subsec:3-conclusion}\indent

    В данной главе рассмотрены основные архитектурные решения реализации проекта, описаны основные классы, которые использовались при написании кода.
    Также, разобрана система сборки каждого из проектов.

    \section{Заключение}\label{sec:closing}\indent

В данной дипломной работе разработано приложение, предназначенное для проведения собеседований.
Разработанное приложение позволяет пользователям осуществлять видео/аудио коммуникацию.
Приложение также предоставляет пользователям возможность обмена текстовыми сообщениями, а также предоставляет редактор кода в режиме онлайн.

Для достижения цели дипломной работы были решены следующие задачи.
\begin{enumerate}
    \item Проанализирована предметная область, связанная с подбором персонала и организацией интервью.
    \item Спроектированы модели функций, модели данных, а также модели потоков данных.
    \item Выбраны средства разработки и приведено обоснование данного выбора.
    \item Разработана клиентская и серверная части веб-приложения.
    \item Проанализирован и разработан пользовательский интерфейс с учетом основных требований и принципов, повышающих его удобство.
\end{enumerate}

Предлагаемая разработка является актуальной, так как решения, представленные на рынке, обладают недостаточным функционалом.
Простота доступа к данным, их поиск, а также простота организации интервью являются важными параметрами удобного для конечного пользователя сервиса.
Разработанное приложение будет актуальным для людей и компаний заинтересованных в поиске вакансий или же поиске новых сотрудников.


    \begingroup
\begin{thebibliography}{00}


\bibitem{api}
    Интерфес прикладного программирования
    [Электронный ресурс] //
    Stud ref
    Режим доступа: \url{https://studref.com/328789/informatika/interfeys_prikladnogo_programmirovaniya} -
    Дата доступа: 24.05.2020
    
\bibitem{smpt}
    Протоколы электронной почты
    [Электронный ресурс] //
    Учебно-методические материалы для студентов кафедры АСОУИ
    Режим доступа: \url{http://www.4stud.info/networking/smtp-pop3-imap.html} -
    Дата доступа: 24.05.2020
    
\bibitem{jvm}
    Java Virtual Machine
    [Электронный ресурс] //
    Википедия. Свободная энциклопедия
    Режим доступа: \url{https://ru.wikipedia.org/wiki/Java_Virtual_Machine} -
    Дата доступа: 24.05.2020

\bibitem{srp}
    Мартин, Р.
    Чистая архитектура. Искусство разработки программного обеспечения. /
    Р. Мартин 
    СПб.:Питер, 2018 -- С. 85.
    % https://kitobz.com/upload/%D0%9C%D0%B0%D1%80%D1%82%D0%B8%D0%BD%20%D0%A0%D0%BE%D0%B1%D0%B5%D1%80%D1%82%20%D0%A7%D0%B8%D1%81%D1%82%D0%B0%D1%8F%20%D0%B0%D1%80%D1%85%D0%B8%D1%82%D0%B5%D0%BA%D1%82%D1%83%D1%80%D0%B0.%20%D0%98%D1%81%D0%BA%D1%83%D1%81%D1%81%D1%82%D0%B2%D0%BE%20%D1%80%D0%B0%D0%B7%D1%80%D0%B0%D0%B1%D0%BE%D1%82%D0%BA%D0%B8%20%D0%BF%D1%80%D0%BE%D0%B3%D1%80%D0%B0%D0%BC%D0%BC%D0%BD%D0%BE%D0%B3%D0%BE%20%D0%BE%D0%B1%D0%B5%D1%81%D0%BF%D0%B5%D1%87%D0%B5%D0%BD%D0%B8%D1%8F-%20www.kitobz.com.pdf

\bibitem{jpa}
    Собеседоваие по Java EE --- Java Persistence API
    [Электронный ресурс] //
    Java study
    Режим доступа: \url{http://javastudy.ru/interview/jpa-questions-answers} -
    Дата доступа: 24.05.2020

\bibitem{http}
    HTTP
    [Электронный ресурс] //
    MDN web docs
    Режим доступа: \url{https://developer.mozilla.org/ru/docs/Web/HTTP} -
    Дата доступа: 24.05.2020

\bibitem{client-server}
    Понятие клиент-серверных систем
    [Электронный ресурс] //
    StudFiles
    Режим доступа: \url{https://studfile.net/preview/7596141/page:3} -
    Дата доступа: 24.05.2020

\bibitem{kotlin}
    Dmitry Jemerov.
    Kotlin in Action /
    Dmitry Jemerov, Svetlana Isakova
    2016 -- C. 6.
    % http://sd.blackball.lv/library/Kotlin_in_Action_(2017).pdf

\bibitem{web-technologies}
    Выбор и описание программных средств и среды разработки реализации сайта
    [Электронный ресурс] //
    StudBooks
    Режим доступа: \url{https://studbooks.net/1998719/informatika/vybor_opisanie_programmnyh_sredstv_sredy_razrabotki_realizatsii_sayta} -
    Дата доступа: 24.05.2020

\bibitem{web-angular}
    AngularJS
    [Электронный ресурс] //
    Википедия. Свободная энциклопедия.
    Режим доступа: \url{https://ru.wikipedia.org/wiki/AngularJS} -
    Дата доступа: 24.05.2020

\bibitem{web-typescript}
    Typescript
    [Электронный ресурс] //
    Википедия. Свободная энциклопедия.
    Режим доступа: \url{https://ru.wikipedia.org/wiki/TypeScript} -
    Дата доступа: 24.05.2020

\end{thebibliography}
\endgroup

\clearpage

    \addition{rest-database}{Схема БД} {
  \addcontentsline{toc}{section}{Приложение}
  \addimghere{rest-database}{1}{}
  \vfill
}
\addition{web-authservice}{Листинг AuthorizationService веб-клиента} {
  \begin{lstlisting}[
  language=TypeScript,
  label={lis:web-authservice}
]
import {Injectable} from '@angular/core';
import {
  HttpClient,
  HttpErrorResponse,
  HttpEvent,
  HttpHandler,
  HttpInterceptor,
  HttpRequest
} from "@angular/common/http";
import {BehaviorSubject, Observable} from "rxjs";
import {LocalStorageService} from "angular-2-local-storage";
import {TokenResponse} from "../model/token-response";
import {catchError, filter, finalize, switchMap, take, tap} from "rxjs/operators";
import {RegistrationModel} from "../model/registration-model";
import {Api} from "../constants";

@Injectable({
  providedIn: 'root'
})
export class AuthorizationService implements HttpInterceptor {

  private REGISTR_URL = Api.REGISTER;
  private TOKEN_URL = Api.TOKEN;

  private ACCESS_TOKEN = "access_token";
  private REFRESH_TOKEN = "refresh_token";
  private EXPIRES = "expires";
  private GRANT_TYPE = "grant_type";
  private USERNAME = "username";
  private PASSWORD = "password";
  private CLIENT_ID = "client_id";
  private CLIENT_SECRET = "client_secret";
  private CLIENT_ID_VALUE = "web";
  private CLIENT_SECRET_VALUE = "secret";
  private SCOPE = "scope";
  private SCOPE_READ = "read";

  private isRefreshingToken: boolean = false;
  private tokenSubject: BehaviorSubject<string> = new BehaviorSubject<string>(null);

  constructor(
    private http: HttpClient,
    private localStorage: LocalStorageService,
  ) {
  }

  get bearerToken(): string {
    return this.localStorage.get(this.ACCESS_TOKEN);
  }

  get refreshToken(): string {
    return this.localStorage.get(this.REFRESH_TOKEN)
  }

  intercept(request: HttpRequest<any>, next: HttpHandler): Observable<HttpEvent<any>> {
    const bearerToken = this.bearerToken;
    let req = this.appendTokenToRequest(request, bearerToken);

    if (bearerToken) {
      req = this.appendTokenToRequest(request, bearerToken);
    } else {
      req = request;
    }

    return next.handle(req).pipe(
      catchError(err => {
        if (err instanceof HttpErrorResponse) {
          switch ((<HttpErrorResponse>err).status) {
            case 401:
              return this.handle401Error(request, next);
            default: {
              throw  err;
            }
          }
        } else {
          throw err;
        }
      })
    )
  }

  login(username: string, password: string): Observable<TokenResponse> {
    return this.getToken(username, password);
  }

  logout() {
    this.cleanTokenData();
  }

  isAuthenticated(): boolean {
    if (this.bearerToken) {
      return true;
    } else {
      return false;
    }
  }

  register(registrationModel: RegistrationModel) {
    return this.http.post(this.REGISTR_URL, registrationModel)
  }

  private handle401Error(request: HttpRequest<any>, next: HttpHandler): Observable<any> {
    if (!this.isRefreshingToken && this.refreshToken) {
      this.isRefreshingToken = true;
      this.tokenSubject.next(null);

      return this.refreshAccessToken().pipe(
        switchMap((token: TokenResponse) => {
          if (token) {
            this.tokenSubject.next(token.access_token);
            this.saveTokenData(token);

            return next.handle(this.appendTokenToRequest(request, token.access_token));
          }
        }),
        catchError(() => {
          this.cleanTokenData();
          return next.handle(request);
        }),
        finalize(() => {
          this.isRefreshingToken = false;
        })
      )
    } else {
      this.isRefreshingToken = false;

      return this.tokenSubject
        .pipe(filter(token => token != null),
          take(1),
          switchMap(token => {
            return next.handle(this.appendTokenToRequest(request, token))
          })
        )
    }
  }

  private getToken(username: string, password: string): Observable<TokenResponse> {
    const url = this.TOKEN_URL;
    const body = new FormData();

    this.appendClientAuthData(body);
    body.append(this.GRANT_TYPE, this.PASSWORD);
    body.append(this.SCOPE, this.SCOPE_READ);
    body.append(this.USERNAME, username);
    body.append(this.PASSWORD, password);

    return this.http.post<TokenResponse>(url, body)
      .pipe(
        tap(success => {
          console.log("success" + success.access_token);
          this.saveTokenData(success);
        })
    );
  }

  private refreshAccessToken(): Observable<TokenResponse> {
    const refreshToken: string = this.refreshToken;
    const url = this.TOKEN_URL;
    const body = new FormData();

    this.appendClientAuthData(body);
    body.append(this.GRANT_TYPE, this.REFRESH_TOKEN);
    body.append(this.REFRESH_TOKEN, refreshToken);

    return this.http.post<TokenResponse>(url, body)
  }

  private saveTokenData(tokenData: TokenResponse) {
    const currentTime = Date.now();
    const expires = currentTime + (tokenData.expires_in * 1000);

    this.localStorage.set(this.ACCESS_TOKEN, tokenData.access_token);
    this.localStorage.set(this.REFRESH_TOKEN, tokenData.refresh_token);
    this.localStorage.set(this.EXPIRES, expires);
  }

  private cleanTokenData() {
    this.localStorage.remove(
      this.ACCESS_TOKEN,
      this.REFRESH_TOKEN,
      this.EXPIRES
    )
  }

  private appendClientAuthData(data: FormData) {
    data.append(this.CLIENT_ID, this.CLIENT_ID_VALUE);
    data.append(this.CLIENT_SECRET, this.CLIENT_SECRET_VALUE);
  }

  private appendTokenToRequest(request: HttpRequest<any>, token: string) {
    return request.clone({
      setHeaders: {
        Authorization: `Bearer ${token}`
      }
    });
  }
}
\end{lstlisting}
}
\addition{web-mgr-product-service}{Листинг MgrProductService веб-клиента} {
  \begin{lstlisting}[
  language=TypeScript,
  label={lis:web-mgr-product-service}
]
import {Injectable} from '@angular/core';
import {HttpClient, HttpParams} from "@angular/common/http";
import {Api} from "../../constants";
import {Product} from "../../model/product";
import {Observable} from "rxjs";
import {ProductDetails} from "../../model/product-details";

@Injectable({
  providedIn: 'root'
})
export class MgrProductService {

  private BASE_URL = Api.MGR_PRODUCTS;
  private DELETED_PATH = "/deleted";
  private PUBLISHED_PATH = "/published";

  constructor(
  private http: HttpClient
  ) { }

  getByCategoryId(id: number): Observable<Product[]> {
    const url = this.BASE_URL;
    const options = {
      params: new HttpParams().set('categoryId', id.toString())
    };

    return this.http.get<Product[]>(url, options)
  }

  getDetailsById(id: number): Observable<ProductDetails> {
    const url = this.BASE_URL + "/" + id;

    return this.http.get<ProductDetails>(url)
  }

  toggleDeleted(id: number): Observable<boolean> {
    const url = this.BASE_URL + this.DELETED_PATH;

    return this.http.put<boolean>(url, id)
  }

  togglePublihsed(id: number) {
    const url = this.BASE_URL + this.PUBLISHED_PATH;

    return this.http.put<boolean>(url, id)
  }

  update(productDetails: ProductDetails): Observable<ProductDetails> {
    const url = this.BASE_URL;

    return this.http.put<ProductDetails>(url, productDetails);
  }
}

\end{lstlisting}
}
\addition{web-product-add}{Листинг ProductAdd компонента веб-клиента} {
  \begin{lstlisting}[
  language=HTML5,
  label={lis:web-product-add-component}
]
<mat-horizontal-stepper [linear]="true"
                        (selectionChange)="onSelectionChange($event)">
    <mat-step label="User primary data"
              [stepControl]="primaryDataForm">
        <form [formGroup]="primaryDataForm" class="container">
            <mat-form-field>
                <input matInput
                       placeholder="Unp"
                       formControlName="unp">
                <mat-error *ngIf="!isValidFormField('unp')">
                    Unp must contain exactly 9 numbers
                </mat-error>
            </mat-form-field>

            <mat-form-field>
                <input matInput
                       placeholder="email"
                       formControlName="email">
            </mat-form-field>

            <mat-form-field>
                <input matInput
                       placeholder="phone"
                       formControlName="phone">
                <span matSuffix>руб.</span>
                <mat-error *ngIf="!isValidFormField('phone')">
                    Enter valid float value from 0.0 to 9999.0
                </mat-error>
            </mat-form-field>

            <mat-form-field>
                <input matInput
                       placeholder="name"
                       formControlName="name">
                <span matSuffix>шт.</span>
                <mat-error *ngIf="!isValidFormField('name')">
                    Enter valid int value from 1 to 9999
                </mat-error>
            </mat-form-field>

            <mat-form-field>
                <input matInput
                       placeholder="name"
                       formControlName="name">
                <span matSuffix>шт.</span>
                <mat-error *ngIf="!isValidFormField('name')">
                    Enter valid int value from 1 to 9999
                </mat-error>
            </mat-form-field>
            <button color="accent"
                    mat-raised-button
                    matStepperNext
                    [disabled]="primaryDataForm.invalid">
                Next
            </button>

        </form>
    </mat-step>

    <mat-step label="Product colors"
              [stepControl]="colorsDataForm">
        <form [formGroup]="colorsDataForm" class="container">
            <div class="two_cols">
                <mat-checkbox *ngFor="let color of colorsArray; let i = index" [formControlName]="i">
                    {{color.article}} - {{color.title}}
                </mat-checkbox>
            </div>
            <div>
                <button color="accent" mat-raised-button matStepperPrevious>Back</button>
                <button color="accent" mat-raised-button matStepperNext>Next</button>
            </div>
        </form>
    </mat-step>

    <mat-step label="Product options data"
              [stepControl]="optionsDataForm">
        <form [formGroup]="optionsDataForm" class="container">
            <ng-container *ngFor="let option of optionArray; let i = index;"
                          [ngSwitch]="option.type">
                <mat-slide-toggle *ngSwitchCase="OptionType.BOOL"
                                  [formControlName]="i">
                    {{option.title}}
                </mat-slide-toggle>

                <mat-form-field *ngSwitchCase="OptionType.INT">
                    <input matInput
                           formControlName="{{i}}"
                           [placeholder]="option.title">
                    <span *ngIf="option.unit" matSuffix>{{option.unit}}</span>
                </mat-form-field>

                <mat-form-field *ngSwitchCase="OptionType.FLOAT">
                    <input matInput
                           formControlName="{{i}}"
                           [placeholder]="option.title">
                    <span *ngIf="option.unit" matSuffix>{{option.unit}}</span>
                </mat-form-field>

                <mat-form-field *ngSwitchDefault>
                    <input matInput
                           formControlName="{{i}}"
                           [placeholder]="option.title">
                    <span *ngIf="option.unit" matSuffix>{{option.unit}}</span>
                </mat-form-field>
            </ng-container>

            <div>
                <button color="accent" mat-raised-button matStepperPrevious>Back</button>
                <button color="accent" mat-raised-button matStepperNext
                        [disabled]="optionsDataForm.invalid">Next
                </button>
            </div>
        </form>
    </mat-step>

    <mat-step label="Check and submit">
        <div class="container">
            <mat-card>
                <mat-card-title>{{checkPrimaryLine}}</mat-card-title>
                <mat-card-subtitle>{{checkSecondaryLine}}</mat-card-subtitle>
                <mat-card-content>
                    <mat-divider></mat-divider>

                    <table class="check_table">
                        <tr>
                            <td>In package:</td>
                            <td>{{productModel.packing | count}}</td>
                        </tr>
                        <tr>
                            <td>Price:</td>
                            <td>{{productModel.price | price}}</td>
                        </tr>
                        <tr>
                            <td>Is price for pack:</td>
                            <td>{{productModel.priceForPack}}</td>
                        </tr>
                        <tr>
                            <td>Package price:</td>
                            <td>{{productModel | price}}</td>
                        </tr>
                    </table>

                    <mat-divider></mat-divider>

                    <table class="check_table">
                        <ng-container *ngFor="let option of productModel.options">
                            <tr>
                                <td>{{option.title}}</td>
                                <td>{{option.value}} {{option.unit}}</td>
                            </tr>
                        </ng-container>
                    </table>

                    <mat-divider></mat-divider>

                    <mat-list *ngIf="productModel.colors && productModel.colors.length > 0">
                        <mat-list-item *ngFor="let color of productModel.colors">
                            {{color.article}} {{color.title}}
                        </mat-list-item>
                    </mat-list>

                </mat-card-content>
            </mat-card>

            <div>
                <button color="accent" mat-raised-button matStepperPrevious>Back</button>
                <button color="accent" mat-raised-button matStepperNext
                        (click)="onStepComplete(StepType.CHECK)"
                        [disabled]="(primaryDataForm.invalid || optionsDataForm.invalid)">Submit
                </button>
            </div>
        </div>
    </mat-step>
</mat-horizontal-stepper>
\end{lstlisting}
}
\addition{web-product-add-css}{Листинг CSS для ProductAdd компонента веб-клиента} {
  \begin{lstlisting}[
  language=CSS,
  label={lis:web-product-add-css}
]
.container {
  display: grid;
  grid-template-columns: 1fr 2fr 1fr;
  grid-auto-rows: minmax(55px, auto);
  align-items: center;
}

.container > * {
  grid-column-start: 2;
  grid-column-end: 3;
  vertical-align: center;
}

.container > .two_cols {
  display: grid;
  grid-template-columns: repeat(2, 1fr);
  grid-column-gap: 20px;
  grid-auto-rows: minmax(40px, auto);
}

.container > :last-child {
  display: grid;
  grid-template-columns: repeat(2, 1fr);
  grid-column-gap: 20px;
}

button {
  width: 100%;
}

.check_table {
  padding: 10px 0;
}

.check_table tr {
  padding: 0;
  line-height: 1.5;
}

.check_table tr td:first-child {
  padding-right: 25px;
}

.mat-horizontal-stepper-header {
  pointer-events: none !important;
}

@media only screen and (max-width: 700px) {
  .container {
    grid-template-columns: 1fr;
  }

  .container * {
    grid-column-start: 1;
    grid-column-end: 2;
  }
}
\end{lstlisting}
}
\addition{android-auth-interceptor} {Листинг AuthTokenInterceptor Android-клиента} {
  \begin{lstlisting}[
  language=Kotlin,
  label={lis:android-auth-interceptor}
]
package com.github.swalffy.magnat_manager.utils.networking.interceptor

import com.github.swalffy.magnat_manager.features.login.RefreshTokenUsecase
import com.github.swalffy.magnat_manager.utils.*
import com.github.swalffy.magnat_manager.utils.common.SharedPrefs
import okhttp3.Interceptor
import okhttp3.Response
import org.koin.core.KoinComponent
import org.koin.core.inject

class AuthTokenInterceptor(
    private val refreshTokenUsecase: RefreshTokenUsecase
) : Interceptor, KoinComponent {

    private val preferences: SharedPrefs by inject()

    override fun intercept(chain: Interceptor.Chain): Response {
        val newRequest = chain.request().newBuilder().apply {
            preferences.bearerToken?.let {
                addHeader("Authorization", "Bearer $it")
            }
        }.build()

        var response = chain.proceed(newRequest)
        if (response.code() == 401) {
            preferences.bearerToken = null
            val refreshToken = preferences.refreshToken

            if (refreshToken?.isNotEmpty() == true) {
                when (tryRestoreSession(refreshToken)) {
                    is Success -> {
                        newRequest.newBuilder().run {
                            preferences.bearerToken?.let {
                                addHeader("Authorization", "Bearer $it")
                            }
                            build()
                        }.let { response = chain.proceed(it) }
                    }
                    is Error -> {
                        preferences.refreshToken = null
                        preferences.bearerToken = null
                    }
                }
            }
        }
        return response
    }

    private fun tryRestoreSession(refreshToken: String) =
        refreshTokenUsecase.performTokenRefresh(refreshToken)
}
\end{lstlisting}
}
\addition{android-request-coroutine}{Листинг ProductListGetAllUseCase Android-клиента} {
  \begin{lstlisting}[
  language=Kotlin,
  label={lis:android-auth-interceptor}
]
package com.github.swalffy.magnat_manager.features.products

import com.github.swalffy.magnat_manager.utils.Error
import com.github.swalffy.magnat_manager.utils.Result
import com.github.swalffy.magnat_manager.utils.Success
import kotlinx.coroutines.Dispatchers
import kotlinx.coroutines.coroutineScope
import kotlinx.coroutines.withContext

class ProductListGetAllUseCase(
    val productsRepository: ProductListRepository
) {

    suspend fun loadProducts(categoryId: Long): Result<List<ProductModel>> {
        return coroutineScope {
            val products = withContext(Dispatchers.IO) {
                productsRepository.getAllProducts(categoryId = categoryId)
            }

            if (products?.isNotEmpty() == true) {
                products.map {
                    ProductModel(
                        id = it.id,
                        title = it.title,
                        article = it.article,
                        deleted = it.deleted,
                        published = it.published,
                        price = if (it.priceForPack)
                            it.price * it.packing
                        else
                            it.price
                    )
                }.let { Success(it) }
            } else {
                Error(RuntimeException("No products"))
            }
        }
    }
}
\end{lstlisting}
}
\addition{android-http-request-manager}{Листинг HttpRequestManager Android-клиента} {
  \begin{lstlisting}[
  language=Kotlin,
  label={lis:android-http-request-manager}
]
package com.github.swalffy.magnat_manager.utils.networking.core

import android.util.Log
import com.github.swalffy.magnat_manager.utils.common.hashOf
import com.github.swalffy.magnat_manager.utils.database.entity.network.NetworkRequestCacheDao
import com.github.swalffy.magnat_manager.utils.database.entity.network.NetworkRequestCacheRecord
import com.github.swalffy.magnat_manager.utils.networking.core.RequestMethod.BodyType
import com.github.swalffy.magnat_manager.utils.networking.interceptor.AuthTokenInterceptor
import com.github.swalffy.magnat_manager.utils.networking.model.FormDataModel
import kotlinx.coroutines.Dispatchers
import kotlinx.coroutines.async
import kotlinx.coroutines.withContext
import okhttp3.*
import okhttp3.HttpUrl.Companion.toHttpUrlOrNull
import okhttp3.MediaType.Companion.toMediaTypeOrNull
import okhttp3.RequestBody.Companion.toRequestBody
import java.io.InputStream
import java.util.concurrent.TimeUnit

private const val HTTP_CALL_TAG = "sw_HTTP_CALL"

class HttpRequestManager(
    private val requestCacheDao: NetworkRequestCacheDao,
    private val networkConfig: NetworkClientConfig,
    private val jsonConverter: JsonConverter
) {

    private val client: OkHttpClient = OkHttpClient.Builder()
        .addInterceptor(AuthTokenInterceptor())
        .callTimeout(networkConfig.timeoutMillis, TimeUnit.MILLISECONDS)
        .build()

    suspend fun <T> request(
        method: RequestMethod,
        path: String,
        queryParams: Map<String, Any?>? = null,
        cacheControl: CacheControl<T>? = null,
        onSuccess: ((InputStream) -> T)? = null,
        onError: ((Int, InputStream?) -> T?)? = { code, _ ->
            Log.w(HTTP_CALL_TAG, "HttpRequestManager request: some error in call $method $path with $code")

            null
        },
        onException: ((Throwable) -> T)? = { throw it }
    ): T? = withContext(Dispatchers.IO) {
        val requestStartTime = System.currentTimeMillis()

        val clearExpiredJob = async { clearExpiredRequests(requestStartTime) }

        val request = buildRequest(
            method = method,
            path = path,
            queryParams = queryParams
        )

        if (cacheControl?.isForceRequest == false) {
            clearExpiredJob.await()

            val cachedRecord = requestCacheDao.getByRequestHash(hashOf(method, request.url.toString()))

            if (cachedRecord != null && cachedRecord.expires > requestStartTime) {
                Log.d(HTTP_CALL_TAG, "HttpRequestManager request: $method $path already cached(expire in ${cachedRecord.expires - cacheControl.expiration})")
                return@withContext null
            }
        }

        runCatching {
            client.newCall(request)
                .execute().use { response ->
                    Log.d(HTTP_CALL_TAG, "HttpRequestManager request: $method $path complete with ${response.code}")

                    if (response.isSuccessful) {
                        cacheControl?.let { cache ->
                            NetworkRequestCacheRecord(
                                id = hashOf(method.toString(), request.url),
                                url = request.url.toString(),
                                method = method.toString(),
                                paramsHash = method.hashCode(),
                                expires = requestStartTime + cache.expiration
                            ).let(requestCacheDao::insert)
                        }

                        response.body?.byteStream()
                            ?.let { stream ->
                                onSuccess?.invoke(stream)
                                    ?.also { cacheControl?.onCache?.invoke(it) }
                            } ?: onError?.invoke(-1, null)
                    } else {
                        onError?.invoke(response.code, response.body?.byteStream())
                    }
                }
        }.getOrElse {
            Log.w(HTTP_CALL_TAG, "HttpRequestManager request: ", it)
            onException?.invoke(it)
        }
    }

    private fun buildRequest(
        method: RequestMethod,
        path: String,
        queryParams: Map<String, Any?>? = null
    ): Request {
        val url = networkConfig.host.toHttpUrlOrNull()
            ?.newBuilder()
            ?.addPathSegments(path)
            ?.also {
                queryParams?.forEach { (key, value) ->
                    it.addEncodedQueryParameter(key, value.toString())
                }
            }?.build()
            ?: error("Can't build url. Base[${networkConfig.host}] path: [$path]")

        val body = (method as? RequestMethod.RequestWithBody<*>)
            ?.let { requestWithBody ->
                when (requestWithBody.bodyType) {
                    BodyType.JSON -> jsonConverter.toJson(requestWithBody.body)
                        .toRequestBody("application/json; charset=utf-8".toMediaTypeOrNull())

                    BodyType.FORM_DATA -> (requestWithBody.body as? FormDataModel)
                        ?.asMap
                        ?.let { bodyMap ->
                            MultipartBody.Builder()
                                .setType(MultipartBody.FORM)
                                .also {
                                    bodyMap.forEach { (key, value) -> it.addFormDataPart(key, value) }
                                }.build()
                        } ?: throw error("Form data body should be Map<String, String>")
                }
            }

        return Request.Builder()
            .method(method.toString(), body)
            .url(url)
            .build()
    }

    private suspend fun clearExpiredRequests(currentTime: Long) {
        requestCacheDao.dropExpired(currentTime)
    }
}
\end{lstlisting}
}
\addition{android-recyclerview-adapter}{Листинг ProductListRecyclerAdapter Android-клиента} {
  \begin{lstlisting}[
  language=Kotlin,
  label={lis:android-recyclerview-adapter}
]
package com.github.swalffy.magnat_manager.features.products

import android.view.View.*
import android.view.ViewGroup
import androidx.recyclerview.widget.RecyclerView
import kotlinx.android.extensions.LayoutContainer
import kotlinx.android.synthetic.main.list_item_product.*

class ProductListRecyclerAdapter(
    private val onClick: (Long) -> Unit
) : RecyclerView.Adapter<ProductListRecyclerAdapter.ProductListItemHolder>() {

    private val items = mutableListOf<ProductModel>()

    override fun onCreateViewHolder(parent: ViewGroup, viewType: Int): ProductListItemHolder {
        val inflatedView = parent.inflate(R.layout.list_item_product, false)
        return ProductListItemHolder(inflatedView)
    }

    override fun getItemCount(): Int = items.size

    override fun onBindViewHolder(holder: ProductListItemHolder, position: Int) {
        val product = items[holder.adapterPosition]
        holder.bind(product)
    }

    fun cleanAddAll(newItems: List<ProductModel>) {
        items.clear()
        items.addAll(newItems)
        notifyDataSetChanged()
    }

    inner class ProductListItemHolder(
        override val containerView: View
    ) : RecyclerView.ViewHolder(containerView), LayoutContainer {

        init {
            containerView.setOnClickListener {
                val productItem = items[adapterPosition]

                onClick.invoke(productItem.id)
            }
        }

        fun bind(product: ProductModel) {
            text_title.text = product.title
            text_article.text = product.article
            view_deleted.visibility = if (product.deleted) VISIBLE else INVISIBLE
            view_published.visibility = if (product.published) INVISIBLE else VISIBLE
            text_price.text = product.price.toString()
        }
    }
}
\end{lstlisting}
}
    
\end{document}